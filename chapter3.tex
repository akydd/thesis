\chapter{Main results}

\section{Introduction}

Let a semitopological semigroup $G$ have a left action on a Hausdorff space $X$.
We begin this chapter by defining translations, invariance, introversion and
amenability of subspaces of $\ell^\infty(X)$ with respect to the left action of $G$ on $X$.
These definitions allow us to define actions that are analogues to the first and second
Arens products, and that reduce to the first and second Arens products when $X = G$.
In Section~\ref{subspaces} we introduce a number of $G$-invariant subspaces of $CB(X)$,
We discuss some properties of each subspace; in particular, we examine
the conditions of invariance and introvertibility.
The focus of Section~\ref{main:arens} is the space $LUC(G,X)$, where $X$
is a locally compact space.
We consider the action of the Banach algebra $LUC(G)^*$ on $LUC(G,X)^*$.
Using ideas of Lau~\cite{lau:arens} and Wong~\cite{wong:arens},
we prove that when $G$ and $X$ are locally compact, the measure
algebra $\mathcal{M}(G)$ is a subset of the topological
center of the action of $LUC(G)^*$ on $LUC(G,X)^*$.
In Section~\ref{main:fairchild} we show that for discrete $G$ and $X$, the set of almost $G$-periodic
points of $\beta X$ is exactly the set of elements in $\beta X$ which belong to some
$G$-minimal set.  We also show that under certain conditions, there exist non almost
$G$-periodic points in $\beta X$ which are in the support of some $G$-invariant measure
of $\beta X$.

\section{Invariance, introversion, and the Arens\\action}\label{main:intro}
Let a semitopological semigroup $G$ have a left action on a nonempty set $X$.  In the case that 
$X$ has its own semigroup structure
(that is. an associative binary operation, $X\times X \rightarrow X$), it is convenient to distinguish
between translations of a function $f$ on $X$ by elements in $G$ and translations of $f$ by elements in $X$.
With this in mind, 
we denote the left translation of a function $f\in \ell^\infty (X)$ by an element $a\in G$ by
\[
%\begin{equation}\label{left}
\lambda_a f(x) = f(ax)
%\end{equation}
\]
for all $x\in X$. We can regard $\lambda_a$ as an operator from $\ell^\infty(X)$ into $\ell^\infty(X)$.
We also define an analogue to the right translation defined
in (\ref{translations}) as follows: for $f\in \ell^\infty(X)$ and $x\in X$, define the function
\[
%\begin{equation}\label{right}
\rho_x f(a) = f(ax),
%\end{equation}
\]
for all $a\in G$, noting that $\rho_x f$ is a function of $G$.  Both functions are well defined.
We can regard $\rho_x$ as an operator from $\ell^\infty(X)$ into $\ell^\infty(G)$.
We define the left and right orbits of $f\in \ell^\infty(X)$ with respect to $G$ by
\[
O_\lambda f = \{ \lambda_a f: a\in G \},\qquad O_\rho f = \{ \rho_x f: x\in X \},
\]
noting that $O_\rho f$ is a subset of $\ell^\infty(G)$.

Let $G$ have a jointly continuous left action on a topological Hausdorff space $X$.

\begin{defn}
Consider a subspace of $CB(X)$ that is defined purely in terms of the action of $G$ on $X$
and the topologies of $G$ and $X$ (i.e. defined independently of the algebraic structure of $X$, if any).
Hereafter, we will always write such a subspace as $Y(G,X)$.  We define the
{\it incident space of $Y(G,X)$} to be $Y(G,G)$, the subspace
% other names for this space?  mutual/intermutial/paired/parallel/incident/twin
of $CB(G)$ defined by the same conditions of $Y(G,X)$, with $X$ replaced by $G$.
For convenience, we write $Y(G)$ for $Y(G,G)$.
\end{defn}

The notion of the incident space allows us describe an analogue
of right invariance within the framework of the left action from $G$.

A linear subspace $Y \subseteq \ell^\infty(X)$ is called {\it left $G$ invariant} if
$O_\lambda f \subseteq Y$ for all $f\in Y$.
A subspace $Y(G,X)$ of $CB(X)$ is called {\it right $X$ invariant} if
$O_\rho f \subseteq Y(G)$ for all $f\in Y(G,X)$.
$Y(G,X)$ is {\it invariant} if it is both left $G$ and right $X$ invariant.

We define functions similar to those in (\ref{introversions}),
and use them to define analogues of left and right introversion.
Let $Y(G,X)$ be a norm closed, left $G$ invariant subspace of $CB(X)$, and
let $n\in Y(G,X)^*$, the dual space of $Y(G,X)$.
$Y(G,X)$ is called {\it left $G$ introverted}
if the function
\[
%\begin{equation}\label{left-i}
n_\lambda f(g) = n(\lambda_g f)
%\end{equation}
\]
is in $Y(G)$ for all $f\in Y$.
When $Y(G,X)$ is a norm closed, right $X$ invariant subspace and $m\in Y(G)^*$,
$Y(G,X)$ is called {\it right $X$ introverted}
if the function
\[
%\begin{equation}\label{right-i}
m_\rho f(x) = m(\rho_x f)
%\end{equation}
\]
is in $Y(G,X)$ for all $f\in Y(G,X)$.
$Y(G,X)$ is {\it introverted} if it is both left $G$ and right $X$ introverted.


We have the following useful lemma concerning the above functions $n_\lambda f$ and $m_\rho f$.
\begin{lemma}\label{conv}
Let $Y(G,X)$ be a norm closed invariant linear subspace of $CB(X)$ containing the constant functions.
For every $f\in Y(G,X)$,
\begin{enumerate}[(i)]
\item $\{n_\lambda f : n\in Y(G,X)^*,\,\|n\| \leq 1 \} = \ol{\cconv\{\rho_x f: x\in X\}}^p$
\item $\{m_\rho f : m\in Y(G)^*,\,\|m\| \leq 1 \} = \ol{\cconv\{\lambda_g f: g\in G\}}^p$
\end{enumerate}
where the closures are taken in the pointwise topologies of the respective function spaces.
\end{lemma}
\proof
(i) Let $f\in Y(G,X)$, $n\in \ball(Y(G,X)^*)$.  By Remark~\ref{dense}(b), we may take a net
$\{n_\alpha\} \subset Y(G,X)^*$ of linear combinations of
point evaluations $\sum_{i=1}^k \xi_i \delta_{x_i}$, with $\sum_{i=1}^k | \xi_i | \leq 1$, such that
$n_\alpha \rightarrow n$ weak*.  For $g\in G$, write $n_\lambda f(g)$ as
\[
\lim_\alpha \left(\sum_{i=1}^k \xi_i \delta_{x_i}\right)_\alpha (\lambda_g f) =
%\lim_\alpha \sum_{i=1}^k \langle (\xi_i \delta_{x_i})_\alpha, \lambda_g f\rangle =
\lim_\alpha \left(\sum_{i=1}^k \xi_i \rho_{x_i}\right)_\alpha f(g).
\]
The proof of (ii) is similar.
\done

\begin{remark}
(a) We will say that $Y(G,X)$ is left or right invariant [resp. introverted], with the understanding
that we refer to left $G$ or right $X$ invariance [resp. introversion].
(b) When $X$ has its own associative binary operation and $X\neq G$, do not confuse our definition
of right $X$ invariance with the usual definition of right invariance for 
a subspace of $\ell^\infty(X)$.
\end{remark}

The following definition of a $G$ invariant mean on subspaces of
$\ell^\infty(X)$ is due to Greenleaf, who first studied
such means when $G$ is a locally compact group (see~\cite{gl:aa}).
\begin{defn}\label{aa}
Let $Y$ be a left $G$-invariant, norm closed subspace of $\ell^\infty(X)$ containing the constant
functions.  A mean $m\in Y^*$ is said to be a {\it left $G$-invariant mean} [GLIM] if
$m(\lambda_g f) = m(f)$ for all $g\in G$ and $f\in Y$.  We denote the set of all such means by $GLIM(Y)$.
$X$ is said to be {\it left amenable} if there exists a GLIM on $\ell^\infty(X)$.
\end{defn}

The new notions of introvertibility allow us to define analogues of
the first and second Arens product as introduced by Arens in~\cite{arens}.
\begin{defn}
Let $Y_1(G,X)$ be a left introverted subspace of $CB(X)$.
For $m\in Y_1(G)^*$, $n\in Y_1(G,X)^*$ and $f\in Y_1(G,X)$, the {\it left Arens action}
of $Y_1(G)^*$ on $Y_1(G,X)^*$ is defined by the map
\[
(m,n) \mapsto m\odot n,
\]
where
\[
m\odot n(f) = \langle m, n_\lambda f\rangle.
\]
Let $Y_2(G,X)$ be a right introverted subspace of $CB(X)$, let
$m\in Y_2(G)^*$, $n\in Y_2(G,X)^*$ and $f\in Y_2(G,X)$.  We similarly define
the {\it right Arens action} of $Y_2(G)^*$ on $Y_2(G,X)^*$ by
\[
n\dotbox m(f) = \langle n, m_\rho f\rangle.
\]
\end{defn}

It is easy to see that the left and right Arens actions are each weak*-weak*
continuous in the first variable.

\begin{proposition}\label{arens:G-inv}
Let $Y(G,X)$ be a left introverted subspace of $CB(X)$.
The left Arens action of $Y(G)^*$ on $Y(G,X)^*$ has the following
properties:
\begin{enumerate}[(i)]
\item If $Y(G,X)$ has a GLIM $n$, then for every $m\in M(Y(G))$, $m\odot n = n$.
\item $Y(G,X)^*$ is a left Banach-$Y(G)^*$ module.
\item For any $g\in G$ and $x\in X$, $\delta_g\odot\delta_x = \delta_{gx}$.
\end{enumerate}
\end{proposition}
\proof
(i) Let $n\in GLIM(Y(G,X))$.  Then $n_\lambda f(g) = n(f)$ for all $g\in G$ and $f\in Y(G,X)$.  Hence
\[
m \odot n(f) = \langle m,n_\lambda f\rangle = \langle m, n(f)\cdot 1\rangle = n(f)
\]
for all $m\in M(Y(G))$.

(ii) $Y(G)^*$ is a Banach algebra with the first Arens product, $\oodot$.
For $l,m \in Y(G)^*$, $n\in Y(G,X)^*$, $f\in Y(G,X)$ and $g$, $h\in G$,
\[
\lambda_{gh}f = \lambda_h(\lambda_g f),
\]
\[
\ell_g(n_\lambda f) = n_\lambda(\lambda_g f),
\]
and so
\begin{align}\label{quickie}
(m_\ell (n_\lambda f))(g) &= \langle m,\;\ell_g(n_\lambda f)\rangle = \langle m,n_\lambda (\lambda_g f)\rangle
= m\odot n(\lambda_g f)\\
&= (m\odot n)_\lambda f(g). \notag
\end{align}
Thus
\begin{align*}
\left(l \oodot m\right)\odot n(f) &= l\oodot m\,(n_\lambda f) = \langle l, m_\ell (n_\lambda f)\rangle
\stackrel{(\ref{quickie})}{=} \langle l, (m\odot n)_\lambda f\rangle \\
&= l \odot \left( m \odot n\right)(f).
\end{align*}
Furthermore,
\begin{align*}
\|m\odot n\| &= \sup_{ \| f \| =1}|m(n_\lambda f)| \leq \|m\| \sup_{ \| f \| =1} \|n_\lambda f\|
= \|m\| \sup_{ \| f \| =1} \sup_{a\in G} |n(\lambda_a f)| \\ &\leq \|m\| \|n\|.
\end{align*}

(iii) For $g\in G$, $x\in X$ and $f\in Y(G,X)$,
\[
\delta_g\odot\delta_x(f) = \langle\delta_g, (\delta_x)_\lambda f\rangle = f(gx) = \delta_{gx}(f).
\]
\done

\begin{defn}
Let $Y(G,X)$ be a left introverted subspace of $CB(X)$.
We define the {\it topological center} of $Y(G,X)^*$ to be the set
\[
Z_Y = \{m\in Y(G)^*:\text{ the map }n\mapsto m\odot n\text{ is w*-w* continuous on }Y(G,X)^*\}.
\]
\end{defn}

\section{Invariant subspaces of $CB(X)$}\label{subspaces}
When a semitopological semigroup $G$ has a jointly continuous left action on
a Hausdorff space $X$, we are able to define the almost periodic, weakly almost
periodic, and left uniformly continuous function spaces on $X$.  We examine
the properties of these function spaces.

\begin{defn}
A function $f\in CB(X)$ is {\it almost periodic} 
if $O_\lambda f$ is relatively compact in the norm topology of $CB(X)$.  We
denote the space of all such functions $AP(G,X)$.
\end{defn}

\begin{lemma}\label{sk4.2.2}
$AP(G,X)$ is an invariant $C^*$-subalgebra of $CB(X)$.
\end{lemma}
\proof
Clearly we have left invariance, since $\{\lambda_g f : g\in G \} =
\{\lambda_a (\lambda_g f) : g\in G \}$ for all $f\in CB(X)$ and $a\in G$.

Let $f\in AP(G,X)$ and $x\in X$.  To see that $\rho_x f \in AP(G)$, start with sequences
$\{g_{n_k}\} \subset \{g_n\} \subset G$ such that
$\|\lambda_{g_{n_k}}f - F\| \rightarrow 0$ for some $F\in CB(X)$.  Then,
\[
\|\ell_{g_{n_k}} (\rho_x f) - \rho_x F\| = \sup_{z\in G\cdot x} |\lambda_{g_{n_k}} f(z) - F(z)|
\leq \|\lambda_{g_{n_k}}f - F\| \rightarrow 0,
\]
showing that $AP(G,X)$ is right invariant.

Take $f_1$, $f_2 \in AP(G)$, a sequence $\{g_n\} \subset G$, a subsequence $\{g_{n_k}\}$ of $\{g_n\}$,
and $F_1$, $F_2 \in CB(X)$ such that
\[
\lim_k \|\lambda_{g_{n_k}}f_i - F_i\| = 0,\qquad i=1,2.
\]
Then,
\[
\|\lambda_{g_{n_k}}(f_1 + f_2) - (F_1 + F_2)\| \leq \sum_{i=1,2}\|\lambda_{g_{n_k}}f_i - F_i\| \rightarrow 0,
\]
and
\[
\|\lambda_{g_{n_k}}(f_1\cdot f_2) - (F_1\cdot F_2)\|
\leq \|\lambda_{g_{n_k}}f_1 - F_1\|\|f_2\| + \|\lambda_{g_{n_k}}f_2 - F_2\|\|F_1\| \rightarrow 0.
\]
Thus, $f_1 + f_2$ and $f_1\cdot f_2 \in AP(G,X)$.  Consequently, $AP(G,X)$ is a subalgebra of $CB(X)$.

To see that $AP(G,X)$ is norm closed, take $\{f_n\} \subset A(G,X)$ such that $\|f_n - f\| \rightarrow 0$ for
some $f\in CB(X)$.  Let $\{g_m\}$ be a sequence in $G$.  For every $m$ and $n$, the function
$\lambda_{g_m}f_n$ is in $AP(G,X)$.
Using the diagonal process, we find a subsequence $\{g_{m_k}\}$ of $\{g_m\}$ such that
for every $i$,
\begin{equation}\label{ap1}
\lim_k \|\lambda_{g_{m_k}}f_i - F_i\| = 0
\end{equation}
for some $F_i \in CB(X)$.
Now, for $j > i$ and large values of $k$,
\begin{align*}
\|F_i - F_j\| &\leq \|F_j - \lambda_{g_{m_k}}f_j\| + \|F_i - \lambda_{g_{m_k}}f_j\| \\
&\leq \|F_j - \lambda_{g_{m_k}}f_j\| + \|F_i - \lambda_{g_{m_k}}f_i\| + \|\lambda_{g_{m_k}}(f_i - f_j)\| \\
&\!\stackrel{(\ref{ap1})}{\leq}\|f_i - f_j\|.
\end{align*}
Thus there exists $F\in CB(X)$ such that $F_n \rightarrow F$ uniformly.  We claim that
$\|\lambda_{g_{m_k}}f - F\| \rightarrow 0$.  Indeed, for every $i$,
\[
\|\lambda_{g_{m_k}}f - F \| \leq \|\lambda_{g_{m_k}}f_i - F_i\| + \|\lambda_{g_{m_k}}f_i - \lambda_{g_{m_k}}f\|
+ \|F_i - F \|,
\]
and so, for large values of $i$,
\[
\lim_k \|\lambda_{g_{m_k}}f - F\| \leq \lim_k \|\lambda_{g_{m_k}}f_i - F_i\| \stackrel{(\ref{ap1})}{=} 0.
\]
Thus $f\in AP(G,X)$.
\done

\begin{theorem}
Let $Y(G,X)$ be a norm closed, involution closed, translation invariant, linear subspace of
$CB(X)$ containing the constant functions.  For any $f\in Y(G,X)$, the following are equivalent:
\begin{enumerate}[(i)]
\item $f\in AP(G,X)$.
\item $\{\rho_x f: x\in X\}$ is relatively norm compact in $CB(G)$.
\item The map $m\mapsto m_\rho f$, $\ball(Y(G)^*)\rightarrow \ell^\infty (X)$ is weak*-norm continuous.
\item The map $n\mapsto n_\lambda f$, $\ball(Y(G,X)^*)\rightarrow \ell^\infty (G)$ is weak*-norm continuous.
\item For every $m\in Y(G,X)^*$ and $n\in Y(G)^*$, the function $m_\rho f$ is in $Y(G,X)$, the function $n_\lambda f$ is in $Y(G)$, $m\odot n(f) = n\dotbox m(f)$, and the map $(m,n) \mapsto m\odot n(f)$ is weak* continuous on $\ball(Y(G)^*) \times \ball(Y(G,X)^*)$.
\end{enumerate}
\end{theorem}
\proof
(i) $\Leftrightarrow$ (iii) Define a map $V_1:\ball(Y(G)^*)\rightarrow \ell^\infty (X)$ by
\[
V_1(m) = m_\rho f.
\]
$V_1$ is weak*-pointwise continuous.  By Lemma~\ref{conv},
\[
V_1(\ball(Y(G)^*)) = \ol{\cconv\{\lambda_g f: g\in G\}}^p.
\]
If (i) holds, then by Mazur's Theorem~\cite[V.2.6]{d&s}, $\ol{\cconv\{\lambda_g f: g\in G\}}^{\|\cdot\|}$ is compact.
Thus the norm and pointwise topologies coincide on $V_1(\ball(Y(G)^*))$, and $V_1$ is weak*-norm
continuous.  If we assume (iii), then $\{m_\rho f: \|m\| \leq 1\}$ is norm compact.  But by Lemma~\ref{conv},
\[
\ol{\{\lambda_g f: g\in G\}}^{\|\cdot\|} \subset \ol{\{m_\rho f: \|m\| \leq 1\}}^{\|\cdot\|},
\]
showing that $f\in AP(G,X)$.

Similarly, (ii) $\Leftrightarrow$ (iv).  We define $V_2:\ball(Y(G,X)^*)\rightarrow \ell^\infty(G)$ by
\[
V_2(n) = n_\lambda f.
\]
$V_2$ is weak*-pointwise continuous.  By Lemma~\ref{conv},
\[
V_2(\ball(Y(G,X)^*)) = \ol{\cconv\{\rho_x f: x\in X\}}^p.
\]
If we assume (ii) then, $\ol{\cconv\{\rho_x f: x\in X\}}^{\|\cdot\|}$ is compact,
and the norm and pointwise topologies coincide on $V_2(\ball(Y(G,X)^*))$.  If we assume (iv),
then $\{n_\lambda f: \|n\| \leq 1\}$ is norm compact.  Relative norm compactness of
$\{\rho_x f: x\in X\}$ follows from noticing that
\[
\ol{\{\rho_x f: x\in X\}}^{\|\cdot\|} \subset \ol{\{n_\lambda f: \|n\| \leq 1 \}}^{\|\cdot\|}.
\]

(v) $\Rightarrow$ (i) Let $f\in Y(G,X)$.  Define a map $W_1: \ball(Y(G)^*) \rightarrow CB(\ball(Y(G,X)^*))$ by
\[
[W_1(m)](n) = m\odot n(f).
\]
$W_1$ is weak*-norm continuous by~\cite[Lemma B.3]{milnes}.  Then $\{W_1(\delta_g):g\in G\}$ is
relatively norm compact, as $\{\delta_g:g\in G\} \subset \ball(Y(G)^*)$.  Notice that
\[
\lambda_g f(x) = \delta_g \odot \delta_x (f)
\]
for all $x\in X$, and so
\[
\{\lambda_g f: g\in G\} = \{[W_1(\delta_g)] \circ \delta: g\in G\},
\]
which is a relatively norm compact set in $CB(X)$.

To prove (v) $\Rightarrow$ (ii),  let $f\in Y(G,X)$.    We define a map
$W_2: \ball(Y(G)^*)\rightarrow CB(\ball(Y(G)^*))$ by
\[
[W_2(n)](m) = n\dotbox m(f).
\]
The proof continuous in the same manner as the proof of (v) $\Rightarrow$ (i).
$W_2$ is weak*-norm continuous, and the set $\{W_2(\delta_x):x\in X\}$ is relatively norm compact.
\[
\rho_x f(g) = \delta_x \dotbox \delta_g (f)
\]
for all $g\in G$, and thus
\[
\{\rho_x f: x\in X\} = \{[W_2(\delta_x)]\circ \delta: x\in X\},
\]
which is a relatively norm compact set in $CB(G)$.

(ii) $\Rightarrow$ (v)  Let $f\in Y(G,X)$.  Recall that from Lemma~\ref{conv}
and the equivalence of (ii) and (iv),
\[
\{n_\lambda f: \|n\| \leq 1\} = \ol{\cconv\{\rho_x f: x\in X\}}^{\|\cdot\|} \subset Y(G).
\]
Assuming (ii), and since $Y(G,X)^* = \ol{\Span\{\delta_x: x\in X\}}^{w^*}$, $n_\lambda f \in Y(G)$
for all $n\in Y(G,X)^*$.  Next, we claim that the map
$(m,n) \mapsto m\odot n(f)$ is continuous for all $m\in\ball(Y(G)^*)$, $n\in\ball(Y(G,X)^*)$ and
$f\in Y(G,X)$.  Indeed, let $\{m_\alpha\}\subset \ball(Y(G)^*)$ and $\{n_\beta\} \subset \ball(Y(G,X)^*)$
such that $m_\alpha \rightarrow m$ and $n_\beta \rightarrow n$, both weak*.  By (iv),
\begin{align*}
|m\odot n (f) - m_\alpha \odot n_\beta (f)| 
%&\leq |\langle m, n_\lambda f\rangle - \langle m, (n_\beta)_\lambda f \rangle|
%+ |\langle m, (n_\beta)_\lambda f \rangle - \langle m_\alpha, (n_\beta)_\lambda f\rangle | \\
&\leq \|m\|\|n_\lambda f - (n_\beta)_\lambda f\| + \|m - m_\alpha\|\|(n_\beta)_\lambda f\| \rightarrow 0.
\end{align*}
To see that $m_\rho f \in Y(G,X)$, notice that the continuity of the map $(m,n)\mapsto m\odot n(f)$
is all we need to show that
$\{\lambda_g f: g\in G\}$ is relatively norm compact (see the proof of (v) $\Rightarrow$ (i)), and
then by (iv), the map $V_1$ is weak*-norm continuous.
As already seen in the proof of (i) $\Rightarrow$ (iii),
\[
\{m_\rho f: \|m\|\leq 1\} = \ol{\cconv\{\lambda_g f: g\in G\}}^{\|\cdot\|} \subset Y(G,X).
\]
The fact that $Y(G,X)^* = \ol{\Span\{\delta_x: x\in X\}}^{w^*}$ proves that
$m_\rho f \in Y(G,X)$ for all $m\in Y(G,X)^*$.
It remains to be seen that the map
$(n,m) \mapsto n\dotbox m(f)$ is continuous for all $m\in\ball(Y(G)^*)$, $n\in\ball(Y(G,X)^*)$ and
$f\in Y(G,X)$.  Indeed, let $\{m_\alpha\}\subset \ball(Y(G)^*)$ and $\{n_\beta\} \subset \ball(Y(G,X)^*)$
such that $m_\alpha \rightarrow m$ and $n_\beta \rightarrow n$ in the  weak* topology.
By the weak*-norm continuity of $V_1$,
\begin{align*}
|n\dotbox m (f) - n_\beta \dotbox m_\alpha (f)| &\leq \|n\|\|m_\rho f - (m_\alpha)_\rho f\|
 + \|n-n_\beta\|\|(m_\alpha)_\rho f\| \rightarrow 0.
\end{align*}
Clearly, $\delta_g \odot \delta_x(f) = \delta_x \dotbox \delta_g (f)$ for every $g\in G$ and every $x\in X$.
By the joint continuity of the maps $(m,n) \mapsto m\odot n(f)$ and $(m,n) \mapsto n\dotbox m(f)$,
$m\odot n(f) = n\dotbox m(f)$ for all $m\in \ball(Y(G)^*)$ and $n\in \ball(Y(G,X)^*)$, and thus holds for all
$m\in Y(G)^*$ and $n\in Y(G,X)^*$.

(i) $\Rightarrow$ (v) Let $f\in AP(G,X)$.  By Lemma~\ref{conv} and Mazur,
\[
\{m_\rho f: \|m\| \leq 1\} = \ol{\cconv\{\lambda_g f: g\in G\}}^{\|\cdot\|} \subset Y(G,X).
\]
Assuming (i), and since $Y(G)^* = \ol{\Span\{\delta_g: g\in G\}}^{w^*}$, $m_\rho f \in Y(G,X)$
for all $m\in Y(G)^*$.

Next we show that $n_\lambda f \in Y(G)$ for all $n\in Y(G,X)^*$.
Let $\{m_\alpha\}\subset Y(G)^*$ and $\{n_\beta\}\subset Y(G,X)^*$ such that $m_\alpha \rightarrow m$ and
$n_\beta \rightarrow n$ in the weak* topologies of their respective spaces.  By the continuity of $V_1$,
\[
|n\dotbox m(f) - n_\beta \dotbox m_\alpha (f)| \leq \|n\|\|m_\rho f - (m_\alpha)_\rho f\| + \|n-n_\beta\|\|(m_\alpha)_\rho f\| \rightarrow 0.
\]
As in the proof of (v) $\Rightarrow$ (i), $\{\rho_x f: x\in X\}$ is relatively norm compact.  Then, applying (iv),
Lemma~\ref{conv} and Mazur,
\[
\{n_\lambda f: \|n\|\leq 1\} = \ol{\cconv\{\rho_x f: x\in G\}}^{\|\cdot\|} \subset Y(G).
\]
Once again, using $Y(G,X)^* = \ol{\Span\{\delta_x: x\in X\}}^{w^*}$, we reason that $n_\lambda f \in Y(G)$
for all $n\in Y(G,X)^*$.
By the weak*-norm continuity of $V_2$, the map $(m,n) \mapsto m\odot n(f)$
is continuous on $\ball(Y(G)^*)\times\ball(Y(G,X)^*)$.
Finally, the equality $m\odot n(f) = n\dotbox m(f)$ follows as in the proof of (ii) $\Rightarrow$ (v).
\done

\begin{corollary}
$AP(G,X)$ is the largest involution closed introverted subspace $Y(G,X)$ of $CB(X)$ containing the constant functions
with the following properties:
\begin{enumerate}[(i)]
\item $m\odot n = n\dotbox m$ for all $m\in Y(G)^*$, $n\in Y(G,X)^*$.
\item The mapping $(m,n) \rightarrow m\odot n$ is jointly continuous on norm bounded subsets of $Y(G)^* \times Y(G,X)^*$.
\end{enumerate}
\end{corollary}

\begin{defn}
A function $f\in CB(X)$ is {\it weakly almost periodic} 
if $O_\lambda f$ is relatively compact in the weak topology of $CB(X)$.  We
denote the space of all such functions $WAP(G,X)$.
\end{defn}

\begin{lemma}\label{sk4.2.3}
$WAP(G,X)$ is an invariant $C^*$-subalgebra of $CB(X)$ containing
$AP(G,X)$.
\end{lemma}
\proof
The inclusion $AP(G,X)\subset WAP(G,X)$ is trivial, as norm compactness implies weak compactness.
Left invariance follows from observing that $\{\lambda_a (\lambda_g f) : a\in G\} \subset \{\lambda_a f: a\in G\}$
for all $g\in G$.  To show right invariance, we need only to show that $\lim_i \lim_j \rho_x f(a_i b_j)
= \lim_j \lim_i \rho_x f(a_i b_j)$ for sequences $\{a_i\}$ and $\{b_j\}$ in $G$.  Notice that
$\rho_x f(a_i b_x) = \delta_{b_j x}(\lambda_{a_i}f)$.  $\lambda_{a_i}f$ has the double limit property,
as shown in the next theorem.

Let $f_1$, $f_2 \in WAP(G,X)$.  Choose a sequence $\{g_n\} \subset G$ and a subsequence $\{g_{n_k}\}\subset \{g_n\}$
such that for all $\phi\in CB(X)^*$,
\begin{equation}\label{quick2}
| \phi(\lambda_{g_{n_k}}f_i) -\phi(F_i)|\rightarrow 0,\quad i=1,2,
\end{equation}
where $F_1$, $F_2 \in CB(X)$.  Clearly,
\[
|\phi(\lambda_{g_{n_k}}(f_1 + f_2)) - \phi(F_1 + F_2)| \leq |\phi(\lambda_{g_{n_k}}f_1) - \phi(F_1)|
+ |\phi(\lambda_{g_{n_k}}f_2) - \phi(F_2)| \stackrel{(\ref{quick2})}{\rightarrow} 0,
\]
and so $f_1+f_2 \in WAP(G,X)$.
To see that $f_1\cdot f_2 \in WAP(G,X)$, we identify $CB(X)$ with the space $C(\Omega)$ of continuous complex-valued
functions on the compact spectrum of the unital $C^*$-algebra $CB(X)$.  The weak convergence of
$\lambda_{g_{n_k}}f_1$ and $\lambda_{g_{n_k}}$ to $F_1$ and $F_2$ (respectively) implies the pointwise convergence
of $\lambda_{g_{n_k}}(f_1\cdot f_2)$ to $F_1\cdot F_2$.  Furthermore,
$\|\lambda_{g_{n_k}}(f_1\cdot f_2)\| \leq \|f_1\|\|f_2\|$.
The pointwise convergence and norm boundedness is equivalent to the weak sequential compactness
in $C(\Omega)$ (see~\cite[Theorem 1.3]{eberlein}).  As a result, $f_1\cdot f_2 \in WAP(G,X)$, and so $WAP(G,X)$
is a subalgebra of $CB(X)$.

To see that $WAP(G,X)$ is norm closed, take $\{f_n\} \subset WAP(G,X)$ such that
$\|f_n - f\| \rightarrow 0$ in $CB(X)$.  For a sequence $\{g_m\}\subset G$
use the diagonal process to produce a subsequence $\{g_{m_k}\}$ such that for each $i$
and each $\phi\in CB(X)^*$,
\begin{equation}\label{wap1}
|\phi(\lambda_{g_{m_k}}f_i) - \phi(F_i)| \rightarrow 0
\end{equation}
for some $F_i \in CB(X)$.  For $j>i$,
\begin{align*}
\|F_i - F_j\| &= \sup_{\stackrel{\scriptstyle \|\phi\|\leq 1}{\phi\in CB(X)^*}} |\langle \phi, F_j\rangle - \langle \phi, F_i\rangle |\\ &= \sup_{\stackrel{\scriptstyle \|\phi\|\leq 1}{\phi\in CB(X)^*}} \lim_k |\langle \phi, \lambda_{g_{m_k}}f_j\rangle - \langle \phi, \lambda_{g_{m_k}}f_i\rangle | \\
&= \|f_j - f_i\|.
\end{align*}
This implies that $\{F_n\}$ is a norm Cauchy sequence in $CB(X)$, and as such, $F_n$ is norm convergent to a function
$F\in CB(X)$.  For every $i$ and $\phi\in CB(X)^*$,
\begin{align*}
|\phi(\lambda_{g_{m_k}}f) - \phi(F)| &\leq |\phi(F) - \phi(F_i)| + |\phi(F_i) - \phi(\lambda_{g_{m_k}}f_i)| \\
&\qquad\qquad\quad +|\phi(\lambda_{g_{m_k}}f_i) - \phi(\lambda_{g_{m_k}}f)| \\
&\leq \|\phi\|\|F - F_i\| + |\phi(F_i) - \phi(\lambda_{g_{m_k}}f_i)| + \|\phi\|\|f_i - f\|.
\end{align*}
In particular, for very large $i$,
\[
\lim_k |\phi(\lambda_{g_{m_k}}f) - \phi(F)| \leq |\phi(F_i) - \phi(\lambda_{g_{m_k}}f_i)| \stackrel{(\ref{wap1})}{=} 0,
\]
proving that $WAP(G,X)$ is norm closed in $CB(X)$.
\done

\begin{theorem}
Let $Y(G,X)$ be a norm closed, involution closed, translation invariant, linear subspace of
$CB(X)$ containing the constant functions.  For any $f\in Y(G,X)$, the following are equivalent:
\begin{enumerate}[(i)]
\item $f\in WAP(G,X)$.
\item $\{\rho_x f : x\in X\}$ is relatively weakly compact.
\item $\lim_i \lim_j f(g_i x_j) = \lim_j \lim_i f(g_i x_j)$ for sequences $\{g_i\}\subset G$ and $\{x_j\}\subset X$,  whenever all limits exist.
\item for every $m\in Y(G)^*$ and $n\in Y(G,X)^*$, the function $m_\rho f$ is in $Y(G,X)$, the function $n_\lambda f$ is in $Y(G)$, and $m\odot n(f) = n\dotbox m(f)$.
\end{enumerate}
\end{theorem}
\proof
(i) $\Leftrightarrow$ (iii)  Let $f\in WAP(G,X)$.  In~\cite{groth}, Grothendieck showed that
the relatively weak compactness of $\{\lambda_g f: g\in G\}$
is equivalent to that set having the double limit property in $\{\delta_x : x\in X\}$.  Thus,
for sequences $\{g_i\}\subset G$ and $\{x_j\}\subset X$,
\[
\lim_i \lim_j \delta_{x_j} (\lambda_{g_i} f) = \lim_j \lim_i \delta_{x_j} (\lambda_{g_i} f),
\]
whenever all the limits exist.

(iii) $\Leftrightarrow$ (ii)  Notice that for sequences $\{g_i\}$ and $\{x_j\}$ as above,
\[
f(g_i x_j) = \rho_{x_j} f(g_i) = \delta_{g_i}(\rho_{x_j}f),
\]
showing that $\{\rho_x f: x\in X\}$ has the double limit property in a bounded subset of $CB(G)^*$.  By the double limit theorem,
this is equivalent to $\{\rho_x f: x\in X\}$ being relatively weak compact in $CB(G)$.

(i) $\Rightarrow$ (iv) Let $f\in WAP(G,X)$ and $m\in \ball(Y(G)^*)$.  We will only show that $m_\rho f \in Y(G,X)$.
The proof that $n_\lambda f \in Y(G)$ for $n\in\ball(Y(G,X)^*)$ is similar, so we safely omit the proof.
Define a map $\phi: \ball(Y(G)^*) \rightarrow \ell^\infty(X)$ by
\[
\phi(m) = m_\rho f.
\]
$\phi$ is weak*-pointwise continuous, and by Lemma~\ref{conv}
\[
\phi(\ball(Y(G)^*)) = \ol{\cconv\{\lambda_g f: g\in G\}}^p.
\]
Now, consider a bounded function $F\in \ol{\cconv\{\lambda_g f: g\in G\}}^p$,
\[
F(x) = \lim_\alpha \left(\sum_{i=1}^k \xi_i \lambda_{g_i} \right)_\alpha f(x)
\]
where $\sum_{i=1}^k | \xi_i | \leq 1$, for all $x\in X$.  Then,
\[
\sum_j\zeta_j\delta_{x_j}(F) = \sum_j\zeta_j\delta_{x_j}\left(\lim_\alpha\left(\sum_{i=1}^k \xi_i \lambda_{g_i} \right)_\alpha f \right).
\]
Since $Y(G,X)^* = \ol{\Span\{\delta_x : x\in X\}}^{w^*}$, $F\in \ol{\cconv\{\lambda_g f: g\in G\}}^w$.
Assuming (i) and applying a theorem~\cite[V.6.4]{d&s} of Krein-\u{S}mulian,
\[
\phi(\ball(Y(G)^*)) = \ol{\cconv\{\lambda_g f: g\in G\}}^w
\]
and is weak compact in $Y(G,X)$.  Thus $m_\rho f \in Y(G,X)$ for every $m\in \ball(Y(G)^*)$.

Next we show that $m\odot n(f) = n\dotbox m(f)$ for all $m\in Y(G)^*$ and $n\in Y(G,X)^*$.
The weak topology is stronger than the pointwise topology, and so
the weak and pointwise topologies coincide on $\phi(\ball(Y(G)^*))$.  Thus
$\phi$ is weak*-weak continuous from $\ball(Y(G)^*)$ to $\ell^\infty (X)$.
We use a special case of Grothendieck's completeness theorem~\cite[Proposition A.8]{milnes}
to extend the continuity of $\phi$ to $Y(G)^*$.
For any $n\in Y(G,X)^*$, define a linear functional $\omega_n$ on $Y(G)^*$ by
\[
\omega_n(m) = \langle n, m_\rho f\rangle.
\]
$\omega_n$ is weak* continuous on $\ball(Y(G)^*)$ by the definition of $\phi$.
Next, apply Grothendieck's completeness theorem to show that there exists a function $F_n\in Y(G)$
such that for all $m\in Y(G)^*$,
\[
\omega_n(m) = m(F_n).
\]
Then $\omega_n$ is a linear weak* continuous functional on $Y(G)^*$.
Since $n\in Y(G,X)^*$ was arbitrary, $\phi$ is weak*-weak continuous from
$Y(G)^*$.
%To see that $m\odot n = n\dotbox m$, consider the map
%\[
%\psi(m) = \langle n, m_\rho f \rangle.
%\]
%As $m_\alpha \rightarrow m$, $\psi(m_\alpha) \rightarrow \psi(m)$,
%since $(m_\alpha)_\rho f \rightarrow m_\rho f$ weakly.  Thus $\psi$
%is a weak* continuous linear functional on $Y(G)^*$.  Now $Y(G) \subset Y(G)^{**} \subset (Y(G)^*)^+$
%and is a total subspace as such.  By~\cite[V.3.9]{d&s}, 
%\[
%\psi(m) = m(h_n)
%\]
%where $h_n \in Y(G)$.
We claim that the mapping $n\mapsto F_n$ is weak*-weak continuous and linear.  Indeed, linearity follows
from the linearity of $n$.  If $\{n_\alpha\}\subset Y(G,X)^*$ and $n_\alpha \rightarrow n$ weak*,
then for any $m\in Y(G)^*$,
\[
m(F_{n_\alpha}) = \omega_{n_\alpha}(m) = \langle n_\alpha, m_\rho f \rangle \rightarrow \langle n, m_\rho f\rangle = m(F_n).
\]
We also have
\[
F_{\delta_x}(g) = \delta_g (F_{\delta_x}) = \omega_{\delta_x}(\delta_g) = \langle \delta_x, (\delta_g)_\rho f\rangle = \delta_x \dotbox \delta_g(f).
\]
%\[
%h_{\delta_x}(g) = \delta_x \dotbox \delta_g (f).
%\]
The linear span of the point evaluations is equal to $Y(G,X)^*$, and thus for all $n\in Y(G,X)^*$,
\[
F_n (g) = n\dotbox \delta_g (f).
\]
That is, $F_n = n_\lambda f$, and so
\[
n\dotbox m (f) = \langle n, m_\rho f \rangle = m(F_n) = \langle m, n_\lambda f \rangle = m\odot n (f).
\]

(iv) $\Rightarrow$ (iii) For $f\in Y(G,X)$ and sequences $\{g_i\}\subset G$, $\{x_j\}\subset X$,
\begin{align*}
\lim_i \lim_j f(g_i x_j) &= \lim_i \left(\lim_j \delta_{x_j}\dotbox\delta_{g_i}(f)\right) = \lim_i \delta_x \dotbox \delta_{g_i} (f)
= \lim_i \delta_{g_i} \odot \delta_x (f) \\
&= \delta_g \odot \delta_x (f) = \delta_x \dotbox \delta_g (f)= \lim_j \delta_{x_j} \dotbox \delta_g (f) =\lim_j \delta_g \odot \delta_{x_j}(f)\\
&= \lim_j \left( \lim_i \delta_{g_i} \odot \delta_{x_j} (f) \right) = \lim_j \lim_i f(g_i x_j)
\end{align*}
whenever all the limits exist.
\done

\begin{corollary}
$WAP(G,X)$ is the largest involution closed, invariant, left introverted subspace of $CB(X)$
containing the constant functions that satisfies the following:
\begin{enumerate}[(i)]
\item $Y(G,X)$ is introverted.
\item $m\odot n = n\dotbox m$ for all $m\in Y(G)^*$ and $n\in Y(G,X)^*$.
\item The mapping $(m,n) \mapsto m\odot n$ is separately continuous.
\end{enumerate}
\end{corollary}

\begin{defn}
A function $f\in CB(X)$ is called {\it left uniformly continuous} if the
map $a\mapsto\lambda_a f$ from $G$ into $CB(X)$ is continuous.
We denote the set of all such functions by $LUC(G,X)$.
\end{defn}

\begin{lemma}
$LUC(G,X)$ is an invariant, left introverted $C^*$-subalgebra of $CB(X)$.
\end{lemma}
\proof
Let $\{g_\alpha\}\subset G$ such that $g_\alpha \rightarrow g$ in $G$, and let $f\in LUC(G,X)$.
Left invariance follows from observing that
\[
\|\lambda_{g_\alpha} (\lambda_a f) - \lambda_g (\lambda_a f) \| = \|\lambda_{ag_\alpha}f - \lambda_{ag}f\|
\]
for all $a\in G$.

To show right invariance, fix $x\in X$.  Then
\begin{align*}
\|\ell_{g_\alpha}(\rho_x f) - \ell_g (\rho_x f)\|
&= \sup_{a\in G} |\ell_{g_\alpha}(\rho_x f)(a) - \ell_g (\rho_x f)(a)| \\
&= \sup_{a\in G} |f(g_\alpha ax) - f(gax)| \\
&= \sup_{z\in G\cdot x} |f(g_\alpha z) - f(gz)| \\
&\leq \sup_{y\in X} |f(g_\alpha y) - f(gy)| \rightarrow 0,
\end{align*}
since $G\cdot x \subseteq X$ and $f\in LUC(G,X)$.

$n_\lambda f$ is in $LUC(G)$ for all $n\in LUC(G,X)^*$, since
\[
\|\ell_{g_\alpha} (n_\lambda f) - \ell_g(n_\lambda f)\| =
\sup_{a\in G} |n(\lambda_{{g_\alpha} a}f) - n(\lambda_{ga}f)| \rightarrow 0,
\]
by the weak* continuity of $n$, and by the left uniform continuity of $f$.

To see that $LUC(G,X)$ is a subalgebra of $CB(X)$, let $f_1$, $f_2 \in LUC(G,X)$, and
take $\{g_\alpha\}$ as above.
Then,
\[
\|\lambda_{g_\alpha}(f_1 + f_2) - \lambda_g(f_1 + f_2)\| \leq
\|\lambda_{g_\alpha}f_1 - \lambda_g f_1\| + \|\lambda_{g_\alpha}f_2 - \lambda_g f_2\| \rightarrow 0,
\]
and
\[
\|\lambda_{g_\alpha}(f_1\cdot f_2) - \lambda_g(f_1\cdot f_2)\| \leq
\|f_1\|\|\lambda_{g_\alpha}f_2 - \lambda_g f_2\| + \|f_2\|\|\lambda_{g_\alpha}f_1 - \lambda_g f_1\| \rightarrow 0
\]
by the left uniform continuity of $f_1$ and $f_2$.

Finally, to see that $LUC(G,X)$ is norm closed, take $\{g_\alpha\}$ as above and let $\{f_\beta\}\subset LUC(G,X)$
such that $f_\beta$ converges uniformly to some $F\in CB(X)$.  Then
\[
\|\lambda_{g_\alpha}F - \lambda_g F\| \leq \|\lambda_{g_\alpha}F - \lambda_{g_\alpha}f_\beta\| +
\|\lambda_{g_\alpha}f_\beta - \lambda_g f_\beta\| + \|\lambda_g f_\beta - \lambda_g F\| \rightarrow 0,
\]
by the norm convergence of $f_\beta$ to $F$, and by the left uniform continuity of each $f_\beta$.
Thus $F\in LUC(G,X)$.
\done

We will take a closer look at the Arens action of $LUC(G)^*$ on $LUC(G,X)^*$
in Section~\ref{main:arens}.

\begin{lemma}
$AP(G,X) \subset LUC(G,X)$.
\end{lemma}
\proof
Let $f\in AP(G,X)$.  Clearly the map $g\mapsto \lambda_g f$ is pointwise continuous.  $O_\lambda f$
is relatively norm compact, so the norm and pointwise topologies agree on this set.
\done

\begin{lemma}
Let $G$ be a locally compact group.  Then $WAP(G,X) \subset LUC(G,X)$.
\end{lemma}
\proof
Let $f\in WAP(G,X)$.  We claim that the map $g\mapsto \lambda_g f$, from $G$ to $WAP(G,X)$
is weakly continuous.  Indeed, if the net $\{g_\alpha\}\subset G$ converges to $g\in G$,
then $\langle \delta_x ,\lambda_{g_\alpha}f \rangle \rightarrow \langle \delta_x, \lambda_g f\rangle$.
Therefore $\{\lambda_{g_\alpha}f\}$ has at least one weak cluster point, $\lambda_g f$.  The relative
weak compactness of $O_\lambda f$ implies that $\{\lambda_{g_\alpha}f\}$ converges weakly to $\lambda_g f$.
A result of Lau~\cite[p. 151]{lau:actions} (see also~\cite[Theorem 7]{mitchell} for the proof when $X=G$) shows that the map
$g\mapsto \lambda_g f$ is norm continuous.  For the sake of completeness we give the details of the argument.

Define an action of $G$ on $M(WAP(G,X))$ by:
\[
\langle am, f\rangle = \langle m, \lambda_{a^{-1}} f\rangle,
\]
for $a\in G$, $m\in M(WAP(G,X))$ and $f\in WAP(G,X)$.  The map $(a,m) \mapsto am$ is continuous
when $WAP(G,X)^*$ has the weak* topology.  Indeed, we show that the map is separately continuous.
Let $\{a_\alpha\}\subset G$ such that $a_\alpha \rightarrow a \in G$.
Then
\[
\langle a_\alpha m,f\rangle = \langle m, \lambda_{a_\alpha^{-1}}f\rangle \rightarrow \langle m, \lambda_{a^{-1}} f\rangle
= \langle am, f\rangle,
\]
by the weak continuity of the map $g\mapsto \lambda_g f$.
Let $\{m_\beta\}\subset M(WAP(G,X))$ such that $m_\beta \rightarrow m$ weak* in $M(WAP(G,X))$.
Then
\[
\langle am_\beta, f\rangle = \langle m_\beta, \lambda_{a^{-1}} f \rangle \rightarrow \langle am,f\rangle.
\]
Ellis' result~\cite[Theorem 1]{ellis} implies that the action $(a,m)\mapsto am$ is jointly continuous.
Next, suppose that $f\not\in LUC(G,X)$.  That is, suppose there exists $\epsilon > 0$ and nets $\{a_\alpha\} \subset G$,
$\{x_\alpha\}\subset X$ with $a_\alpha \rightarrow a \in G$ such that
\[
\epsilon \leq |\lambda_{a_\alpha}f(x_\alpha) - \lambda_a f(x_\alpha)|
\]
for each $\alpha$.  Choose a subnet $\{x_{\alpha_\gamma}\}\subset \{x_\alpha\}$ such that
$\delta_{x_{\alpha_\gamma}} \rightarrow m$ weak*.  Then,
\begin{align*}
0<\epsilon &\leq \lim_\gamma |\langle\delta_{x_{\alpha_\gamma}}, \lambda_{a_{\alpha_\gamma}}f\rangle
- \langle\delta_{x_{\alpha_\gamma}}, \lambda_a f\rangle| \\
&=\lim_\gamma |\langle a_{\alpha_\gamma}^{-1} \delta_{x_{\alpha_\gamma}}, f\rangle
- \langle a^{-1}\delta_{x_{\alpha_\gamma}},f\rangle| = 0.
\end{align*}
As a result of this contradiction, $f\in LUC(G,X)$.
\done

\begin{remark}
Easy calculations will show that the space $CB(X)$ is invariant.
\end{remark}

%\begin{lemma}
%The following are equivalent.
%\begin{enumerate}[(i)]
%\item $m\in Z_{AP(G,X)}$
%\item $m \mapsto m\odot n$ is weak*-weak* continuous
%\item $m \mapsto m\odot n$ is weak*-weak* continuous on norm bounded subsets of $AP(G,X)^*$
%\end{enumerate}
%\end{lemma}
%\proof
%The proof of (i) $\Rightarrow$ (ii) is identical to the proof in Lemma~\ref{lau:arens}.

%(ii) $\Rightarrow$ (iii) is trivial.

%(iii) $\Rightarrow$ (i): As in Lemma~\ref{lau:arens}, $m_\rho f \in CB(X)$.  Next, let $g_n$
%be a sequence in $G$.  There exists a subsequence $g_{n_k}$ such that
%$\|\lambda_{g_{n_k}}f - F\| \rightarrow 0$ for $F\in CB(X)$.  Then,
%\begin{align*}
%\|\lambda_{g_{n_k}}m_\rho f - m_\rho F\| &= \sup_{x\in X}|m(\rho_{g_{n_k} x} f) - m(\rho_x F)| \\
%&\leq \|m\| \sup_{x\in X} \sup_{h\in G} |\rho_{g_{n_k} x} f(h) - \rho_x F(h)| \\
%&= \|m\| \sup_{x\in X} \sup_{h\in G} |\lambda_h (\lambda_{g_{n_k}} f(x) - F(x))| \\
%&\leq \|m\| \|\lambda_{g_{n_k}}f - F\| \rightarrow 0.
%\end{align*}
%Thus $m_\rho f \in AP(G,X)$.  As in Lemma~\ref{lau:arens},
%\[
%n \dotbox m = m\odot n
%\]
%for any $n\in AP(G,X)^*$.
%\done

%\begin{remark}
%Lemma~\ref{sk4.2.4} and the first statements in lemmas~\ref{sk4.2.2} and~\ref{sk4.2.3} and
%their proofs are the work of Skantharajah (\cite{skantharajah:msc}, lemmas 4.2.4,
%4.2.2 and 4.2.3 respectivly), who states and proves them for $X= G/H$, where $H$ is a closed
%subgroup of $G$.
%\end{remark}

%It is also an easy exercise to see that both $AP(G,X)^*$ and $WAP(G,X)^*$ are left Banach-$LUC(G)^*$
%modules.
\pagebreak
\section{Arens action of the Banach algebra \\$LUC(G)^*$ on $LUC(G,X)^*$}\label{main:arens}

In this chapter we shall consider the topological center $Z_{LUC}$ of the Arens action
of $LUC(G)^*$ on $LUC(G,X)^*$.
Using ideas of Lau and Wong, we show that if
%Using a technique of Lau, we show that
%$m$ being in the topological center of $LUC(G)^*$ is equivalent to weak*-weak* continuity of the map
%$n \mapsto m\odot n$.  Using a modified result of Wong, we also show that when
$G$ and $X$ are locally compact, then the measure algebra
$\mathcal{M}(G)$ is a subset of $Z_{LUC}$, and that in general $\mathcal{M}(G)\subsetneq Z_{LUC}$.

Let $Z$ denote the set of all $m\in LUC(G)^*$ such that the function $m_\rho f$ is in $LUC(G,X)$
for all $f\in LUC(G,X)$, with $m\odot n = n\dotbox m$ for all $n\in LUC(G)^*$.
We begin by proving the following equivalence:
\begin{lemma}\label{lau:arens}
Let $m \in LUC(G)^*$.  The following are equivalent:
\begin{enumerate}[(i)]
\item $m\in Z$.
%\item The map $n\mapsto m\odot n$ is weak*-weak* continuous.
\item $m\in Z_{LUC}$.
\item The map $n\mapsto m\odot n$ is weak*-weak* continuous on norm bounded subsets of $LUC(G,X)^*$.
\end{enumerate}\end{lemma}

\proof
(i) $\Rightarrow$ (ii) Suppose $m\in Z$, and suppose $\{n_\alpha\}$ is a net in $LUC(G,X)^*$
which converges weak* to $n$.  Then, for all $f \in LUC(G,X)$,
\[
m\odot n_\alpha (f) = n_\alpha \dotbox m(f) \rightarrow n\dotbox m(f) = m \odot n(f)
\]

(ii) $\Rightarrow$ (iii) is trivial.

(iii) $\Rightarrow$ (i) Consider $f \in LUC(G,X)$.  First we establish that $m_\rho f$ is in $CB(X)$.
Let $\{x_\alpha\}$ be a net in $X$
converging to $x$.  The norm bounded net $\{\delta_{x_\alpha}\}$ converges weak* to $\delta_x$ in $LUC(G,X)^*$.
We then have that
\[
m_\rho f(x_\alpha) = m(\rho_{x_{\alpha}}f) = m \odot \delta_{x_\alpha}(f)
\longrightarrow m \odot \delta_x (f) = m_\rho f(x),
\]
thus $m_\rho f \in CB(X)$.

Suppose that $m_\rho f \not\in LUC(G,X)$.  That is, suppose for a net $\{a_\alpha\}$ which converges to $a$ in $G$,
\begin{equation}\label{luc1}
0 < \epsilon \leq \|\lambda_{a_\alpha}(m_\rho f) -\, \lambda_a(m_\rho f) \|.
\end{equation}
Let $\{\theta_\alpha\}\subset \ball(CB(X)^*)$ such that for every $\alpha$,
\begin{equation}\label{luc2}
\langle \theta_\alpha,\lambda_{a_\alpha}(m_\rho f) -\, \lambda_a(m_\rho f) \rangle
= \|\lambda_{a_\alpha}(m_\rho f) -\, \lambda_a(m_\rho f) \|.
\end{equation}
We claim that for $\theta \in CB(X)^*$, $f \in LUC(G,X)$, and $a \in G$,
\[
\langle \theta,\; \lambda_a(m_\rho f) \rangle = \langle m \odot \left(\delta_a \odot \theta\right), f\rangle.
\]
This claim holds for the case  $\theta = \delta_x$, $x\in X$.  Indeed, by Proposition~\ref{arens:G-inv}(iii),
\begin{align*}
\langle\delta_x,\;\lambda_a(m_\rho f) \rangle &= m_\rho f(ax) = m(\rho_{ax}f) = \langle m, (\delta_{ax})_\lambda f\rangle
= m\odot\delta_{ax}(f)\\ &= \langle m\odot(\delta_a \odot \delta_x) ,f\rangle.
\end{align*}
Then, if $\theta \geq 0$, $\|\theta\| = 1$, apply Remark~\ref{dense} to obtain a net of
convex combinations of point evaluations $\theta_\beta = \sum_{i=1}^k \xi_i \delta_{x_i}$, such
that $\theta_\beta \rightarrow \theta$ weak* in $CB(G)^*$.  By (iii),
\begin{align*}
\langle\theta,\; \lambda_a(m_\rho f) \rangle &= \lim_\beta \langle\theta_\beta ,\; \lambda_a(m_\rho f) \rangle =
\lim_\beta \langle m \odot \left(\delta_a \odot \theta_\beta \right), f\rangle \\
&= \langle m \odot \left(\delta_a \odot \theta \right), f\rangle,
\end{align*}
which proves our claim.  From (\ref{luc1}) and (\ref{luc2}), we have that
\begin{align}
\epsilon &\leq \langle \theta_\alpha,\lambda_{a_\alpha}(m_\rho f) -\, \lambda_a(m_\rho f) \rangle
         = |\langle m \odot(\delta_{a_\alpha} \odot \theta_\alpha) - m \odot(\delta_a \odot \theta_\alpha),f \rangle|\notag \\
&\leq |\langle m \odot(\delta_{a_\alpha} \odot \theta_\alpha) - m \odot(\delta_a \odot \theta),f \rangle|\label{luc3}\\
&\qquad\qquad\qquad + |\langle m \odot(\delta_a \odot \theta) - m \odot(\delta_a \odot \theta_\alpha),f \rangle|.\label{luc4}
\end{align}
We may assume that $\theta_\alpha$ converges weak* to $\theta$ in $LUC(G,X)^*$ after passing
to a subnet, if necessary.  Furthermore, the action of $G$ on the unit ball of $LUC(G,X)^*$
defined by $(a,\theta)\mapsto \delta_a \odot \theta$ is jointly continuous.  Indeed,
let $\{a_\sigma\}\subset G$ and $\{\theta_\iota\}\subset LUC(G,X)^*$, $\theta_\iota \geq 1$, $\|\theta_\iota\|=1$
such that $a_\sigma \rightarrow a$ in $G$ and $\theta_\iota \rightarrow \theta$ weak* in $LUC(G,X)^*$.  Then
for all $f\in LUC(G,X)$,
\[
\delta_{a_\sigma} \odot \theta_\iota (f) = \langle \delta_{a_\sigma},(\theta_\iota)_\lambda f\rangle
= \theta_\iota (\lambda_{a_\sigma} f) \rightarrow \theta(\lambda_a f)
= \delta_a \odot \theta(f),
\]
by the definitions of $f$ and $\theta_\iota$.
Assuming (iii) and taking $\alpha$
sufficiently large, (\ref{luc3}) and (\ref{luc4}) are each less than $\epsilon/2$,
which is a contradiction.  Therefore $m_\rho f \in LUC(G,X)$.

To show that $n\dotbox m = m \odot n$,
we first claim that this is true when $n = \delta_x$, $x\in X$.  Indeed, for all $f\in LUC(G,X)$,
\begin{align*}
(\delta_x \dotbox m)(f) &= \langle\delta_x, m_\rho f\rangle = m_\rho f(x) = m(\rho_x f) \\
&= \langle m, (\delta_x )_\lambda f\rangle = (m\odot\delta_x )(f).
\end{align*}
This also holds for convex combinations of point measures.  Applying Remark~\ref{dense}
by writing any $n \in LUC(G,X)^*$ such that $n \geq 0$ and $\|n\|=1$ as the weak* limit of a net
of such combinations, we obtain that for every $f \in LUC(G,X)$,
\begin{align*}
n\dotbox m(f) &= \langle n, m_\rho f \rangle = \lim_\alpha \langle n_\alpha, m_\rho f\rangle \\
&= \lim_\alpha m \odot n_\alpha(f) = m \odot n(f).
\end{align*}
Thus, $m \in Z$.
\done

When $X$ is locally compact, let $\tau_X$ denote the locally convex topology on $\mathcal{M}(X)$
determined by the family of seminorms $\{p_f : f\in LUC(G,X)\}$, where
$p_f (\mu) =| \int f\:d\mu|$, $\mu\in \mathcal{M}(X)$.  Each $\mu\in \mathcal{M}(X)$ defines an element in $LUC(G,X)^*$ via
$\langle \mu, f\rangle = \int f\:d\mu$.
Let $\tau_G$ denote the locally convex topology above for the case $X=G$.

\begin{lemma}\label{wong}
Let $G$ be a locally compact semitopological semigroup with jointly continuous action
on a locally compact Hausdorff space X.
\begin{enumerate}[(i)]
\item For every $\mu\in \mathcal{M}(G)$, the map $n \mapsto {\mu}\odot n$ is weak*-weak* continuous on norm bounded subsets of $LUC(G,X)^*$.
\item For every $n\in LUC(G,X)^*$, the map $\mu\mapsto {\mu}\odot n$ is $\tau_G$-weak* continuous.
\item For $\mu\in \mathcal{M}(G)$, $\nu\in \mathcal{M}(X)$, ${\mu}\odot{\nu}(f) = \langle\mu * \nu, f\rangle$ for all $f\in C_0(X)$.
\item $n\in GLIM(LUC(G,X))$ if and only if ${\mu}\odot n = n$ for any $\mu\in \mathcal{M}_0 (G) = \{\mu\in \mathcal{M}(G) : \mu \geq 0, \|\mu\| = 1\}$.
%\item $LUC(G,X)$ has a $G$-invariant mean iff there is a net $\{\nu_\alpha \}$ in $M_0 (X)$ such that
%$\mu * \nu_\alpha - \nu_\alpha \rightarrow 0$ in the $\tau$ topology of $M(X)$ for any $\mu \in M_0 (G)$.
\end{enumerate}
\end{lemma}

\proof
(i) Consider a net $\{n_\alpha \} \subset LUC(G,X)^*$ such that $n_\alpha \rightarrow n$ weak* in $LUC(G,X)^*$
with $\|n_\alpha \|, \|n\| \leq K$.
%\[
%{\mu}\odot n_\alpha (f) - {\mu}\odot n(f) = \int (n_\alpha)_\lambda f\:d\mu - \int n_\lambda f\:d\mu.
%\]
\begin{align*}
|(n_\alpha)_\lambda f(a) - (n_\alpha)_\lambda f(b)| &= |n_\alpha (\lambda_a f) - n_\alpha (\lambda_b f)| \leq \|n_\alpha \| \|\lambda_a f - \lambda_b f\| \\
&\leq K \|\lambda_a f - \lambda_b f\|
\end{align*}
for all $a,\,b\in G$, which shows that $\{(n_\alpha)_\lambda f\}$ is a equicontinuous family of functions
(\cite[p. 232]{kelley:top}).
We have pointwise convergence of $(n_\alpha)_\lambda f$ to $n_\lambda f$, since
\[
|(n_\alpha)_\lambda f(g) - n_\lambda f(g)| = |n_\alpha (\lambda_g f) - n(\lambda_g f)| \rightarrow 0
\]
by the weak* convergence of $n_\alpha$ to $n$.  By~\cite[Theorem 7.15]{kelley:top}, $(n_\alpha)_\lambda$ converges uniformly
to $n_\lambda f$ on compact subsets of $G$.
Now, for $\mu$ with compact support $U$,
\[
\left|\int(n_\alpha)_\lambda f\:d\mu - \int n_\lambda f\:d\mu \right| \leq
\int \sup_{g\in U}|(n_\alpha)_\lambda f(g) - n_\lambda f(g)|\:d\mu \rightarrow 0.
\]
Such measures are norm dense in $\mathcal{M}(G)$, and
\[
\|(n_\alpha)_\lambda f\| = \sup_{g\in G}|(n_\alpha)_\lambda f(g)| = \sup_{g\in G} |n_\alpha (\lambda_g f)| \leq \|n_\alpha \| \|f\| \leq K\|f\|.
\]
Thus ${\mu} \odot n_\alpha \rightarrow {\mu} \odot n$ weak* for all $\mu\in \mathcal{M}(G)$.

(ii) Consider $n\in LUC(G,X)^*$.  Let $\{\mu_\alpha\}\subset\mathcal{M}(G)$ converge to $\mu$ in the $\tau_G$ topology.
Then for all $f\in LUC(G,X)$,
\[
{\mu}_\alpha \odot n(f) - {\mu}\odot n (f) = \int n_\lambda f\:d\mu_\alpha - \int n_\lambda f\:d\mu \rightarrow 0,
\]
since $n_\lambda f \in LUC(G)$.  Therefore ${\mu}_\alpha \odot n \rightarrow {\mu}\odot n$ weak* in $LUC(G,X)^*$.

(iii) Let $\mu\in\mathcal{M}(G)$ and $\nu\in\mathcal{M}(X)$.  $\mathcal{M}(G)$ has an action on $\mathcal{M}(X)$ defined by
\[
\iint f(gx)\:d\mu(g)d\nu(x) = \int f(gx)\:d(\mu * \nu)(gx),
\]
for $f\in C_0(X)$ which makes $\mathcal{M}(X)$ a Banach module over $\mathcal{M}(G)$.  From~\cite[p. 299]{gl:aa},
$C_0 (X) \subset LUC(G,X)$.
Thus, for $f\in C_0(X)$,
\begin{eqnarray*}
{\mu}\odot{\nu}(f) & = & {\mu}({\nu}_\lambda f) = \int {\nu}_\lambda f(g)\:d\mu(g) = \int{\nu}(\lambda_g f)\:d\mu(g) \\
& = & \iint (\lambda_g f)(x)\:d\nu(x)d\mu(g) = \iint f(gx)\:d\nu(x)d\mu(g) \\
& = & \iint f(gx)\:d\mu(g)d\nu(x) = \int f(gx) d(\mu * \nu)(gx) \\
& = & \langle {\mu * \nu}, f\rangle.
\end{eqnarray*}

(iv) To see ``$\Leftarrow$'', notice that $\mathcal{M}_0 (G)$ contains the point
measures $\delta_g$.  Let $\mu = \delta_g$.  If ${\mu}\odot n = n$,
\[
n(f) = {\delta}_g \odot n(f) = \int n_\lambda f\:d\delta_g = n(\lambda_g f)
\]
for all $g\in G$, which proves that $n$ is $G$-invariant.

``$\Rightarrow$''
For any $\mu\in \mathcal{M}_0(G)$, take a net $\{\mu_\alpha \}$
of convex combinations of point evaluations such that ${\mu}_\alpha$ converges weak* to ${\mu}$ in
$LUC(G)^*$.  Then $\mu_\alpha$ converges to $\mu$ in the $\tau_G$ topology in $\mathcal{M}(G)$.  By (ii),
${\mu}_\alpha \odot n \rightarrow {\mu}\odot n$ weak*.  By Proposition~\ref{arens:G-inv},
${\mu}_\alpha \odot n = n$ since $n$ is $G$-invariant.  Thus ${\mu}\odot n = n$
for all $\mu\in \mathcal{M}_0(G)$.
\done

\begin{theorem}\label{wong2}
Let $G$ be a locally compact semitopological semigroup with jointly continuous action
on a locally compact Hausdorff space X. If $\mu \in \mathcal{M}(G)$, then ${\mu} \in Z_{LUC}$.
\end{theorem}
\proof
By Lemma~\ref{wong} (i), the map ${\mu} \mapsto m\odot {\mu}$ is weak*-weak* continuous on norm bounded subsets
of $LUC(G,X)^*$.  By Lemma~\ref{lau:arens}, ${\mu} \in Z_{LUC}$.
\done

The following result was suggested to me by Lau (personal communication).
\begin{example}
In general, $\mathcal{M}(G) \subsetneq Z_{LUC}$.  Take $G$ to be an infinite semigroup with the discrete
topology and with a product defined by $ab = b$ for all $a$, $b\in G$.  Let $X=G$.
Then $LUC(G) = LUC(G,X) =\ell^\infty (G),$ and $\mathcal{M}(G) = \ell^1(G)$.
Clearly $Z_{LUC}\subseteq \ell^\infty(G)^*$.  To see that $\ell^\infty(G)^* \subseteq Z_{LUC}$,
let $\{n_\alpha\}\subset\ell^\infty(G)^*$ such that $n_\alpha \rightarrow n$ weak*.
Then,
\begin{align*}
\|(n_\alpha)_\ell f - n_\ell f\| &= \sup_{g\in G} |(n_\alpha)_\ell f(g) - n_\ell f(g)|
= \sup_{g\in G} |n_\alpha (\ell_g f) - n(\ell_g f)| \\
&= \sup_{g\in G} |n_\alpha (f) - n(f)| \rightarrow 0,
\end{align*}
and so
\[
m\odot n_\alpha (f) = \langle m, (n_\alpha)_\ell f\rangle \rightarrow \langle m, n_\ell f\rangle
= m\odot n (f).
\]
Thus $Z_{LUC}=\ell^\infty (G)^* = \ell^1(G)^{**}$, while $\mathcal{M}(G) = \ell^1(G)$, a non-reflexive space.
\end{example}

\begin{remark}
(a) Lau
proved that if $G$ is either a locally compact group or a cancellative discrete semigroup
and $X=G$, then $Z_{LUC} = \mathcal{M}(G)$ (see~\cite[Theorem 1]{lau:arens}).
Recently, Neufang~\cite{neufang} used a different technique to prove the same result when $G$ is a locally compact
group.
(b) Lemma~\ref{lau:arens} and its proof are based
on~\cite[Lemma 2]{lau:arens}.  Lemma~\ref{wong} and its proof are based on~\cite[Lemma 3.1]{wong:arens}.
\end{remark}

\section{$G$-minimal sets and supports of $G$-invariant measures}\label{main:fairchild}

Let $G$ be a discrete semigroup with a left action on a discrete space $X$.
In this setting, the spaces $LUC(G)$ and $LUC(G,X)$ are simply $\ell^\infty (G)$
and $\ell^\infty (X)$ respectively.  We then consider the action of
$\ell^\infty (G)^*$ on $\ell^\infty (X)^*$ as defined by the Arens action.

Let $(\kappa,\beta X)$ be the Stone-\u{C}ech compactification of $X$.
For discrete $X$, $\beta X$ is homeomorphic to the character space of $\ell^\infty(X)$.  In other words,
we regard $\beta X$ as the space of all multiplicative
linear functionals $\phi$ on $\ell^\infty(X)$ such that $\phi(1) = \|\phi\| = 1$, with the weak* topology
inherited from $\ell^\infty(X)^*$.  More precisely, for $x\in X$ we identify $\kappa(x)\in\beta X$ with
the evaluation functional $\delta_x \in \ell^\infty(X)^*$.  It is easy to see that for any subset $U\subset X$,
the set $\ol{\{\kappa(x): x \in U\}}^{w^*}$ is weak* closed, and that its complement in open in $\beta X$.
Thus all closed and open subsets of $\beta X$ are of the form $\ol{\{\kappa(x): x\in U\}}^{w^*}$,
for $U\subset X$.  For convenience, we denote the weak* closure of a set $V$ by $\ol{V}$ hereafter.

The action of $G$ on $X$ extends to an action of $\beta G$ on $\beta X$.
For any $a\in G$ and $f_1,f_2 \in \ell^\infty (X)$, $\lambda_a (f_1\cdot f_2) = (\lambda_a f_1 \cdot  \lambda_a f_2)$.
Then
\[
m\odot n(f_1\cdot f_2) = \langle m, n_\lambda (f_1\cdot f_2) \rangle = \langle m, n_\lambda f_1 \cdot n_\lambda f_2 \rangle
= m\odot n(f_1) \cdot m\odot n(f_2)
\]
for any $m\in \beta G$, $n\in \beta X$, which shows that $m\odot n \in \beta X$.
Since $\kappa(G)\subset \beta G$, we may also define an action of $G$ on $\beta X$ by 
\[
(a,n) \mapsto \kappa(a) \odot n
\]
for $a\in G$ and $n\in \beta X$.

\begin{remark}\label{w*-w*}
By Lemma~\ref{lau:arens} and Theorem~\ref{wong2}, the map
\[
n \mapsto \kappa(a) \odot n
\]
is weak*-weak* continuous for all $a\in G$ and $n\in \beta X$.
\end{remark}

For a fixed $n\in \beta X$ we denote the set of all products
\[
\kappa(G) \odot n = \{\kappa(g) \odot n : g\in G\}.
\]
For a fixed $g\in G$ and any subset $U\subset \beta X$  we denote the set
of all products
\[
\kappa(g) \odot U = \{\kappa(g) \odot n: n\in U\}.
\]
If $K\subset G$ and $U\subset \beta X$, we use the following notation:
\[
\{\kappa(K)\}^{-1} \odot U = \{ n\in \beta X : \kappa(k) \odot n \in U\text{ for some }k\in K\}.
\]
If $A\subset G$ and $K$ is as above, we denote
\[
\{K\}^{-1}A = \{g\in G: kg\in A\text{ for some }k\in K\}.
\]
We have an isometric $*$-isomorphism $T$ from $C_c (\beta X) = CB(\beta X)$ onto $\ell^\infty (X)$,
$\tilde{f}\mapsto f$, where
\[
f(x) = \tilde{f}(\kappa(x)),
\]
for $\tilde{f}\in C_c (\beta X)$ and $x\in X$.  By the Reisz representation theorem, we identify
$CB(\beta X)^*$ with $\mathcal{M}(\beta X)$ in the usual way:
\[
\langle T^*n, \tilde{f}\rangle = \int\tilde{f}\:d(T^*n).
\]

\begin{proposition}\label{probmeas}
$n\in GLIM(\ell^\infty(X))$ if and only if
$T^* n$ is a probability measure on $\beta X$ such that
$(T^*n)(\{\kappa(g)\}^{-1} \odot U) = (T^*n)(U)$ for all $g\in G$ and Borel sets $U\subset \beta X$.
\end{proposition}
\proof
``$\Rightarrow$''
Suppose $n\in GLIM(\ell^\infty(X))$.  $T^*n$ is a regular Borel measure for $\beta X$ such that
$\|T^*n\| = 1$.  Let $\tilde{f}\in CB(\beta X)$, $f\geq 0$.  Then
\[
(T \lambda_{\kappa(g)}\tilde{f})(x) = \lambda_{\kappa(g)}\tilde{f}(\kappa(x)) = \tilde{f}(\kappa(gx)) =
(T\tilde{f})(gx) = \lambda_g f(x),
\]
for any $x\in X$ and $g\in G$.
By the invariance of $n$,
\[
\int\:\lambda_{\kappa(g)} \tilde{f}\:d(T^*n) = \langle T^*n, \lambda_{\kappa(g)} \tilde{f}\rangle =
\langle n,\lambda_g f\rangle = \langle n, f\rangle = \langle T^*n, \tilde{f}\rangle = \int \tilde{f}\:d(T^*n).
\]
Since $\tilde{f}\in CB (\beta X)$ was arbitrary,
\[
T^*n(\{\kappa(g)\}^{-1} \odot U) = T^*n(U)
\]
for all Borel sets $U\subset \beta X$.

To see ``$\Leftarrow$'', notice that
\[
\chi_{\{\kappa(g)\}^{-1}\odot U} = \lambda_{\kappa(g)}\chi_U.
\]
for any $g\in G$ and Borel set $U\subset \beta X$.  For any function $f\in\ell^\infty(X)$,
we can approximate $\tilde{f}\in CB(\beta X)$ by simple functions.
As a result,
\[
\langle n, f\rangle = \int\tilde{f}\:d(T^*n) = \int\lambda_{\kappa(g)}\tilde{f}\:d(T^*n) = \langle n, \lambda_g f \rangle.
\]
\done

\begin{defn}
Let $n\in \ell^\infty(X)^*$.  $T^*n$ is called $G$-{\it invariant} if $n \in$ \linebreak $GLIM(\ell^\infty(X))$.
\end{defn}

\begin{defn}
$n \in \beta X$ is called {\it left almost $G$-periodic} if for every neighbourhood $U$ of $n$
there exists a subset $A \subset G$ such that there is a finite subset $K \subset G$ with
$G = \{K\}^{-1}A$ and $\kappa(A) \odot n \subset U$.  We denote the set of all almost
$G$-periodic elements in $\beta X$ by $A^{G,X}$.
\end{defn}

\begin{defn}
A nonempty subset $U$ of $\beta X$ is called $G$-{\it invariant} if $\kappa(g) \odot U \subset U$ for all
$g\in G$.  $U$ is called $G$-{\it minimal} if it is closed and minimal with respect to this property.
We denote the elements of $\beta X$ which belong to a $G$-minimal set by $B^{G,X}$.
\end{defn}

We denote by $K^{G,X}$ the elements in $\beta X$ which are in the support of some $G$-invariant
measure.

\begin{example}
Take $n\in \beta X$.  Clearly, for all $a\in G$,
$\kappa(a) \odot \kappa(G) \odot n \subset \kappa(G) \odot n$.  Thus, $\ol{\kappa(G) \odot n}$
is a closed $G$-invariant set.
\end{example}

\begin{proposition}\label{fairchild2.1}
Let $n\in GLIM(\ell^\infty(X))$.  Then $\supp (T^*n)$ is a \linebreak$G$-invariant set.
\end{proposition}
\proof
Suppose $m \in \supp (T^*n)$ and let $g\in G$.  Let $\ol{\kappa(A)}$ be an open neighbourhood
of $\kappa(g) \odot m$, so that $m\in \{\kappa(g)\}^{-1} \odot \ol{\kappa(A)}$.  By Proposition~\ref{probmeas},
\[
0 < T^*n (\{\kappa(g)\}^{-1} \odot \ol{\kappa(A)}) = T^*n (\ol{\kappa(A)}),
\]
showing that $\ol{\kappa(A)}\subset \supp (T^*n)$.  Consequently $\kappa(g) \odot m \in \supp (T^*n)$.
\done

Let $U$ be a closed subset of $\beta G$.
In the case where $X=G$, it is well known that for all $n\in M(\ell^\infty(G))$,
$\supp(T^*n) \subset U$ if and only if $n\in \ol{\conv U}$~\cite{wilde}.
The same can be said when $X$ is an arbitrary discrete space.

\begin{proposition}~\label{w&w}
Let $n\in M(\ell^\infty(X))$, and let $U$ be a closed subset of $\beta X$.
Then $\supp(T^*n) \subset U$ if and only if $n\in\ol{\conv U}$.
\end{proposition}
\proof
``$\Rightarrow$''  The set $U$ is of the form $\ol{\{\delta_x: x\in V\}}$, where $V\subset X$.
Next, write $n$ as the weak* limit of convex combinations of point evaluations to obtain the formula
\begin{equation}\label{bx1}
\langle T^*n, \chi_U\rangle = \langle n, \chi_V \rangle =
\lim_\alpha\left(\sum_{i=1}^k \xi_i \delta_{x_i}\right)_\alpha(\chi_V),
%= \lim_\alpha\left( \sum_{i=1}^k \xi_i \chi_V(x_i)\right)_\alpha
\end{equation}
But $(\sum_{i=1}^k \xi_i)_\alpha = 1$ for each $\alpha$.  Since $\langle T^*n, \chi_U\rangle = 1$,
$\{\delta_{x_i}\}_\alpha \subset U$ for each $\alpha$.

``$\Leftarrow$''  Notice that for $n = w^*\lim_\alpha\left(\sum_{i=1}^k \xi_i \delta_{x_i}\right)_\alpha \in\ol{\conv U}$,
equation (\ref{bx1}) = 1.
\done

\begin{theorem}\label{fairchild3.1}
$A^{G,X} = B^{G,X}$.
\end{theorem}
\proof
``$\supset$''  Let $U\subset \beta X$ be $G$-minimal, and suppose $m\in U$.
Clearly, $\kappa(g) \odot m \in U$ for all $g\in G$.  i.e., $\kappa(G) \odot w \subset U$.
Since $U$ is closed, $\ol{ \kappa(G) \odot m} \subset U$.
However, since $U$ is $G$-minimal, $U \subset \ol{ \kappa(G) \odot m}$.  Thus 
$U = \ol{\kappa(G) \odot m}$.  Let $V$ be an open neighbourhood of $m$.  Suppose there is some
$v\in U\backslash (\{\kappa(G)\}^{-1} \odot V)$.  Also suppose that $\kappa(g) \odot v \in \{\kappa(G)\}^{-1} \odot V$
for some $g\in G$.  Then there exists some $h\in G$ such that
\[
\kappa(h) \odot \kappa(g) \odot v = \kappa(hg) \odot v \in V,
\]
and so $v\in \{\kappa(hg)\}^{-1} \odot V$,
contradicting $v\not\in \{\kappa(G)\}^{-1} \odot V$.  We have $\kappa(g) \odot v \not\in \{\kappa(G)\}^{-1} \odot V$,
and $\kappa(g) \odot v \in U$ since $U$ is $G$-minimal.  Thus $U\backslash (\{\kappa(G)\}^{-1} \odot V)$
is a closed and $G$-invariant set.  By the $G$-minimality of $U$,
\[
U\cap (\kappa(G)\}^{-1} \odot V) = \emptyset,
\]
i.e. $m\not\in \{\kappa(G)\}^{-1} \odot V$ and
$(\kappa(G) \odot m) \cap V = \emptyset$.  This is impossible since $V$ is open and
$m\in U = \ol{\{\kappa(G)\}^{-1} \odot V}$.  Therefore $U\backslash (\{\kappa(G)\}^{-1} \odot V) = \emptyset$
and $U\subset \{\kappa(G)\}^{-1} \odot V$.  Now, $U$ is closed and thus compact.  So there exists some finite
$K\subset G$ such that
\[
U\subset \{\kappa(K)\}^{-1} \odot V.
\]
Let $A = \{g\in G: \kappa(g) \odot m \in V \}$.
By the definition of $U$, for any $g\in G$ there must exist some $k\in K$ such that
\[
\kappa(k) \odot (\kappa(g) \odot m) = \kappa(kg) \odot m \in V,
\]
giving us that $kg\in A$.  Since $g$
was arbitrary, we have that $G =\{K\}^{-1}A$.  Clearly, $\kappa(A) \odot m \in V$.  Thus $m\in A^{G,X}$.

``$\subset$''  Suppose $m\in A^{G,X}$.  Let $U$ be a $G$-minimal subset of $\ol{\kappa(G) \odot m}$ and
suppose that $m\not\in U$.  Then there exists an open neighbourhood $V$ of $m$ such that
$\ol{V}\cap U = \emptyset$.  Since $m\in A^{G,X}$, there exists $A\subset G$ and finite $K\subset G$
such that $\kappa(A) \odot m \in V$ and $G = \{K\}^{-1}A$.  Now, $k\in K$, $g\in G$ and $kg\in A$
imply that
\[
\kappa(g) \odot m \in \{\kappa(k)\}^{-1} \odot (\kappa(A) \odot m) \subset \{\kappa(k)\}^{-1} \odot V.
\]
Hence
\[
\kappa(G) \odot m \subset \{\kappa(K)\}^{-1} \odot V.
\]
An easy application of Remark~\ref{w*-w*} show that
\[
\ol{\{\kappa(K)\}^{-1} \odot V} \subset \{\kappa(K)\}^{-1} \odot \ol{V}.
\]
Therefore
\[
U\subset\ol{\kappa(G) \odot m} \subset \{\kappa(K)\}^{-1} \odot \ol{V}.
\]
Choosing $k\in K$ such that $(\{\kappa(k)\}^{-1} \odot \ol{V}) \cap U \neq \emptyset$, we see
that for some $n\in\beta X$,
\[
\kappa(k) \odot n \in \ol{V} \cap (\kappa(k) \odot U) \subset \ol{V} \cap U = \emptyset
\]
by the $G$-invariance of $U$.  Since this is impossible, we must have
\[
U \subset (\ol{\kappa(G) \odot m}) \cap U \subset (\{\kappa(K)\}^{-1} \odot \ol{V}) \cap U = \emptyset.
\]
This contradicts $U\neq\emptyset$.  Hence we have that $m\in U$, and we conclude that $m\in B^{G,X}$.
\done

\begin{remark}
The proof for Theorem~\ref{fairchild3.1} is essentially due to Fairchild, who
states the theorem with the additional condition that $G$ be left amenable
(\cite[p. 85]{fairchild}).  The reader will notice that neither of the proofs make use
of an invariant mean.
\end{remark}

\begin{corollary}
When $G$ acts amenably on $X$, $A^{G,X} \subset K^{G,X}$.
\end{corollary}
\proof
Let $n\in GLIM(\ell^\infty(X))$, and let $U$ be a $G$-minimal subset of $\beta X$.  Suppose that $\supp (T^*n) \subset U$.
By Proposition~\ref{fairchild2.1}, $\supp (T^*n)$ is $G$-invariant.  Since $U$ is $G$-minimal,
$\supp (T^*n) = U$.
\done

We now find conditions on $X$ and $G$ that imply $K^{G,X}\backslash A^{G,X} \neq \emptyset$.
We make use of the following definition, lemma and theorem.

\begin{defn}
For $A \subset X$, let
\[
d(A) = \sup\:\{ m(\chi_A) : m\in GLIM(\ell^\infty(X)) \}.
\]
$A$ is called a {\it C-subset} for the pair  $(G,X)$ if $d(A) > 0$ and $d(\{K\}^{-1}A) < 1$ for all
finite $K\subset G$.
\end{defn}

\begin{lemma}\label{fairchild3.3}
Let $A \subset X$, $g\in G$.  Then
\[
\{\kappa(g)\}^{-1} \odot \ol{\kappa(A)} = \ol{ \{\kappa(g)\}^{-1} \odot \kappa(A)} = \ol{\kappa(\{g\}^{-1}A)}.
\]
\end{lemma}
\proof
First take $\kappa(x) \in \kappa(\{g\}^{-1}A)$.  Then $\kappa(g) \odot \kappa(x) = \kappa(gx)$, but $gx \in A$.
Thus
\[
\ol{\kappa(\{g\}^{-1}A)} \subset \ol{ \{\kappa(g)\}^{-1} \odot \kappa(A)}.
\]
Applying Remark~\ref{w*-w*},
\[
\ol{ \{\kappa(g)\}^{-1} \odot \kappa(A)} \subset \{\kappa(g)\}^{-1} \odot \ol{\kappa(A)}.
\]
We now need only show that $\{\kappa(g)\}^{-1} \odot \ol{\kappa(A)} \subset \ol{\kappa(\{g\}^{-1}A)}$.
Take $n\in \{\kappa(g)\}^{-1} \odot \ol{\kappa(A)}$, and let $\kappa(x_\alpha)$ be a net converging to $n$.
By Lemma~\ref{wong},
\[
\kappa(gx_\alpha) = \kappa(g) \odot \kappa(x_\alpha) \rightarrow \kappa(g) \odot n \in \ol{\kappa(A)}.
\]
Since $\ol{\kappa(A)}$ is open in $\beta X$, there exists some $\alpha_0$ such that $\alpha > \alpha_0$
implies $\kappa(gx_\alpha) \in \ol{\kappa(A)}$.  But $\ol{\kappa(A)}\cap\kappa(X) = \kappa(A)$.  Indeed, clearly
$\ol{\kappa(A)}\cap\kappa(X) \supset \kappa(A)$.  Now take some $m\in \ol{\kappa(A)}\cap\kappa(X)$.
Then $m = \kappa(x_0)$ for some $x_0 \in X$, and $m=w^*\lim\kappa(a_\gamma)$ for a net $\{a_\gamma\}\subset A$.
In particular,
\[
|\langle m,\chi_{x_0}\rangle - \langle \kappa(a_\gamma), \chi_{x_0}\rangle | = |\chi_{x_0}(x_0) - \chi_{x_0}(a_\gamma)| \rightarrow 0.
\]
i.e. for large values of $\gamma$, $x_0 = a_\gamma \in A$, showing that $m \in \kappa(A)$.
Hence $\kappa(gx_\alpha) \in \kappa(A)$, and so $x_\alpha \in \{g\}^{-1}A$.
Then
\[
n  = w^* \lim \kappa(x_\alpha) \in \kappa(\{g\}^{-1}A) \subset \ol{\kappa(\{g\}^{-1}A)}.
\]
\done

\begin{theorem}\label{fairchild2.0}
If left amenable $G$ acts on $X$ and $U$ is a closed $G$-invariant
subset of $\beta X$, then for each $n\in U$ and $m\in LIM(\ell^\infty(G))$, $T^*(m\odot n)$
is a $G$-invariant measure with $\supp (T^*(m\odot n)) \subset U$.
\end{theorem}
\proof
To the see that $T^*(m\odot n)$ is $G$-invariant, we need only check that $m\odot n$ is
$G$-invariant.  Let $f\in \ell^\infty (X)$, $g\in G$.  Then,
\[
m\odot n(\lambda_g f) = \langle m, n_\lambda (\lambda_g f) \rangle = \langle m,\,\lambda_g(n_\lambda f) \rangle
= \langle m, n_\lambda f \rangle = m\odot n(f).
\]
Let $\{m_\alpha\}$ be a net of convex combinations of point measures with $m_\alpha \rightarrow m$ weak*.
By the $G$-invariance of $U$, $m_\alpha \odot n \in \conv U$ for each $\alpha$.  But then
\[
m\odot n = \lim_\alpha m_\alpha \odot n \in \ol{\conv U}.
\]
By Proposition~\ref{w&w}, $\supp(T^*(m\odot n)) \subset U$.
\done

\begin{theorem}
If $G$ is left amenable and $(G,X)$ has a C-subset $A$, then $\ol{\kappa(A)}\cap\ol{A^{G,X}} = \emptyset$ and
$\ol{\kappa(A)}\cap K^{G,X} \neq \emptyset$.  Thus $A^{G,X} \subsetneq K^{G,X}$.
\end{theorem}
\proof
First suppose that there exists some $n \in \ol{\kappa(A)}\cap A^{G,X}$.  Notice that $\ol{\kappa(A)}$
is an open set, and so, since $n\in A^{G,X}$, there exists a subset $B\subset G$ and a
finite $K\subset G$ such that $G=\{K\}^{-1}B$ and $\kappa(B) \odot n \subset \ol{\kappa(A)}$.
By Lemma~\ref{fairchild3.3}
\begin{align*}
\kappa(G) \odot n &= \kappa(\{K\}^{-1}B) \odot n \subset \{\kappa(K)\}^{-1} \odot \kappa(B) \odot n\\
& \subset \{\kappa(K)\}^{-1} \odot \ol{\kappa(A)} = \ol{\kappa(\{K\}^{-1}A)},
\end{align*}
which gives us that $\ol{\kappa(G) \odot n} \subset \ol{\kappa(\{K\}^{-1}A)}$.
By Proposition~\ref{arens:G-inv}, for any  $l\in \ol{\kappa(G) \odot n}$ and any $m\in LIM(\ell^\infty(G))$,
$m\odot l \in GLIM(\ell^\infty(X))$.
Apply Theorem~\ref{fairchild2.0} to see that 
\[
\supp(T^*(m\odot l)) \subset \ol{\kappa(G) \odot n} \subset \ol{\kappa(\{K\}^{-1}A)}.
\]
i.e. $T^*(m\odot l) (\chi_{\ol{\kappa(\{K\}^{-1}A)}}) = 1$, contradicting that
$A$ is a C-subset for $(G,X)$.  $\ol{\kappa(A)}$ is open, and so
\[
\ol{\kappa(A)} \cap A^{G,X} = \ol{\kappa(A)} \cap \ol{A^{G,X}} = \emptyset.
\]
Next, notice that for some $\phi \in GLIM(\ell^\infty(X))$,
\[
0 < \phi(\chi_A) = \phi (T \chi_{\kappa(A)}) = (T^*\phi)(\chi_{\kappa(A)}).
\]
Therefore there exists some $a\in A$ such that $\kappa(a)$ is in the support of $T^*\phi$,
a $G$-invariant measure.  We conclude that $\ol{\kappa(A)} \cap K^{G,X} \neq \emptyset$,
\done

We next consider the case when $(G,X)$ has no C-subsets.

\begin{proposition}
Let $G$ be left amenable.  Suppose that the pair $(G,X)$ has no C-subsets.
Then $K^{G,X} \subset \ol{A^{G,X}}$.
\end{proposition}
\proof
Take $n\in K^{G,X}$, and let $U$ be an open neighbourhood of $n$.  There exists a subset
$A\subset X$ such that $n\in \ol{\kappa(A)} \subset U$.  Consequently there exists $\phi\in GLIM(\ell^\infty(X))$
such that $\phi(\chi_A) > 0$.  $A$ is not a C-subset for $(G,X)$, and so there exists a finite
subset $K\subset G$ such that $d(\{K\}^{-1}A) = 1$.  By Remark~\ref{w*-w*} and
Proposition~\ref{probmeas}, there exists $m\in GLIM(\ell^\infty(X))$
such that
\[
\supp(T^*m) \subset \ol{\kappa(\{K\}^{-1}A)} \subset \{\kappa(A)\}^{-1} \odot \ol{\kappa(A)}.
\]
By Proposition~\ref{fairchild2.1}, there exists
a $G$-minimal set $V\subset \{\kappa(K)\}^{-1} \odot \ol{\kappa(A)}$.
Choose $l\in V$ and $k\in K$ such that $\kappa(k)\odot l \in \ol{\kappa(A)} \subset U$.
Then,
\[
\kappa(k)\odot l \in U\cap(\kappa(k)\odot V) = U\cap V
\]
by the $G$-invariance of $V$.  By Theorem~\ref{fairchild3.1},
\[
U\cap B^{G,X} = U\cap A^{G.X} \neq \emptyset.
\]
Since $U$ was arbitrary, $n\in A^{G,X}$.
\done

\begin{remark}
Section~\ref{main:fairchild} is an adaptation of a paper by Fairchild~\cite{fairchild},
in which it is shown that the set of almost periodic points of $\beta G$
is exactly the set of elements which belong to a minimal set, and that
if $(G,G)$ contains a C-subset, then there exist non almost periodic
points of $\beta G$ that are in the support of some invariant measure
for $\beta G$.

\end{remark}