\chapter{Introduction}
A semigroup $G$ together with a Hausdorff topology for which multiplication
in $G$ is separately [resp. jointly] continuous is called a semitopological
[resp. topological] semigroup.
A jointly continuous action of $G$ on a topological Hausdorff space $X$ is
a jointly continuous  map $(a,x) \mapsto ax$ from $G\times X$ into $X$, such that
$(ab)x = a(bx)$ for all $a$, $b\in G$ and $x\in X$, and such that the map 
$x\mapsto ax$, from $X$ 
into $X$, is continuous for each $a\in G$.

Chapter 2 contains basic definitions and results that are used throughout this thesis.
The results are stated without proof.  We also list some well known examples
that demonstrate the ideas contained within this chapter.

It is obvious that multiplication for a topological semigroup $G$ can be
viewed as a jointly continuous action of $G$ onto itself.
It is with this
idea in mind that we develop our ideas in the third chapter.
Under the assumption that a semitopological semigroup $G$ has a jointly continuous
left action on a topological Hausdorff space $X$, we modify the existing definitions
of invariance and introversion for subspaces of $CB(G)$ to obtain similar definitions
for subspaces of $CB(X)$.  These modified definitions allow us to define analogues
to the first and second Arens products.

In Section~\ref{subspaces} we define and examine the properties of several $G$-invariant function spaces
in $CB(X)$ which are analogues of common invariant function spaces in $CB(G)$.
In Section~\ref{main:arens} we examine, in detail,
the action of the dual space of the left uniformly continuous functions on $G$
on the dual space of the left $G$ uniformly continuous functions on $X$.
 
The last section of Chapter 3 is an application of the Arens action.
$\beta G$, the Stone-\u{Cech} compactification of $G$, has an Arens 
action on $\beta X$, the Stone-\u{Cech} compactification of $X$.
We examine the relationship between the almost $G$-periodic points of 
$\beta X$,
the $G$-minimal subsets of $\beta X$, and the elements of $\beta X$ that 
are in the support of a $G$-invariant measure on $\beta X$.

The last chapter of this thesis investigates fixed point properties
of the action of a locally compact group $G$ on a locally convex topological space $X$.
We use Day's fixed point theorem to show the existence of an invariant
measure on coset spaces for certain groups.
We define a fixed point property for the pair $(G:H)$, where $H$ is a closed
subgroup of $G$, such that the
action of $G$ on $X$ need not be jointly continuous.
We prove that this fixed point property is in fact equivalent to a stronger
fixed point property for $(G:H)$ due to Eymard, who requires that the action of $G$ on $X$
be jointly continuous.