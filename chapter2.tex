\chapter{Preliminaries}

\section{Definitions and Notation}

A semigroup $G$ together with a Hausdorff topology for which
multiplication in $G$ is continuous from the left is called a 
{\it right topological semigroup}.
If multiplication in $G$ is separately [resp. jointly] continuous, $G$ is called a
{\it semitopological} [resp. {\it topological}\,] {\it semigroup}.
When
the topology of $G$ is locally compact, $G$ is called a {\it locally compact semitopological}
[resp. {\it topological}\,] {\it semigroup}.
A group $G$ equipped with a Hausdorff topology is called a {\it topological group}
if both multiplication and inversion are continuous.  When the topology of such a group is
locally compact, we say $G$ is a {\it locally compact group}.  It is well known
that a locally compact group $G$ admits a unique left Haar measure; that is, there
exists a unique (up to a constant) regular Borel measure $\nu$ on $G$ such that
$\nu(gE) = \nu(E)$, for every Borel set $E\subset G$ and every $g\in G$.

For an arbitrary topological space $X$, we denote by $X_d$ the space $X$ taken with its discrete topology.
When $G$ is a semigroup [resp. group], we call $G_d$ the {\it discrete semigroup} [resp. {\it group}\,].

Throughout this chapter, $G$ will denote a semitopological semigroup,
unless otherwise stated.

For any nonempty set $X$,
we denote by $\ell^\infty(X)$ the space of all bounded complex-valued functions on $X$.
Then $\ell^\infty(X)$ is a commutative unital $C^*$-algebra with respect to the supremum norm
$\|f\|_\infty = \sup_{x\in X} |f(x)|$, where the multiplication and addition are defined pointwise
and the involution is defined by pointwise complex conjugation, $f^*(x) = \ol{f(x)}$.

For a topological space $X$, there are several important subspaces of $\ell^\infty(X)$ that we will consider.
We denote by $CB(X)$ the space of all bounded, continuous complex-valued functions on $X$.
$C_0 (X)$ denotes all $f\in CB(X)$ such that for every $\epsilon >0$, there exists a compact
subset $K\subset X$ (depending on $f$ and $\epsilon$) such that $|f(x)| < \epsilon$ for every
$x\in X\backslash K$.
%vanishing at infinity (explain this), and
$C_c (X)$ denotes all $f\in CB(X)$ having compact support.  We have the natural inclusion relation
$C_c(X) \subseteq C_0(X) \subseteq CB(X)$, where equality holds when $X$ is compact.
With the operations and topology inherited from $\ell^\infty(X)$, both $CB(X)$ and $C_0(X)$ are $C^*$-subalgebras
of $\ell^\infty(X)$.  Throughout this thesis, we will always consider the space $CB(X)$ to have the
supremum norm topology, unless otherwise stated, noting that $CB(X_d) = \ell^\infty(X)$.

For any function $f\in \ell^\infty(G)$ and $a\in G$, we denote the left and right translations
of $f$ by $a$ by
\begin{equation}\label{translations}
\ell_a f(b) = f(ab), \qquad r_a f(b) = f(ba)
\end{equation}
for all $b\in G$.  We may also view $\ell_a$ and $r_a$ as operators from
$\ell^\infty(G)$ into $\ell^\infty(G)$.
We denote the left and right orbits of $f$ by
\[
O_\ell f = \{ \ell_a f: a\in G \},\qquad O_r f = \{ r_a f: a\in G \}.
\]

A linear subspace $Y \subseteq \ell^\infty(G)$ is called {\it left} [resp. {\it right}\,] {\it invariant} if
$O_\ell f \subseteq Y$ [resp. $O_r f \subseteq Y$] for all $f\in Y$.  $Y$ is said to be {\it invariant}
if it is both left and right invariant.

Let $Y$ be a left [resp. right] invariant norm closed subspace of $\ell^\infty(G)$,
and let $Y^*$ denote the dual space of $Y$.  $Y$ is called {\it left} [resp. {\it right}\,] {\it introverted}
if the function
\begin{equation}\label{introversions}
n_\ell f(g) = n(\ell_g f)\qquad \text{[resp. }n_r f(g) = n(r_g f)\text{]}
\end{equation}
is in $Y$ for each $n\in Y^*$.
$Y$ is called {\it introverted} if it is both left and right introverted.

For any nonempty subset $A$ of a vector space $V$, we denote the
convex hull of $A$ and the convex circled hull of $A$ by $\conv A$ and $\cconv A$, respectively.

Finally, we say that a semitopological semigroup $G$ has a {\it left action} on a topological
space $X$ when there is a map $(a,x) \mapsto ax$ from
$G \times X$ into $X$ such that
\[
(ab)x = a(bx)
\]
for all $a,b\in G$ and $x\in X$, with the map
\[
x\mapsto ax
\]
continuous for each fixed $a\in G$.  Such an action is called {\it jointly continuous}
if the map $(a,x) \mapsto ax$ is jointly continuous.

\section{Subspaces of $CB(G)$}
We give below a brief overview of common subspaces of the $C^*$-algebra $CB(G)$
that are encountered in analysis.  Proofs of the stated results
can be found in Chapter 4 of the useful book on analysis on semigroups
by Berglund {\it et al.}~\cite{milnes}.

A function $f\in CB(G)$ is called {\it almost periodic} if the left
orbit $O_\ell f$ is relatively compact in $CB(G)$.  We denote the set
of all such functions by $AP(G)$.  $AP(G)$ is an
introverted $C^*$-subalgebra of $CB(X)$ containing the constant functions.

A function $f\in CB(G)$ is called {\it weakly almost periodic} if the left
orbit $O_\ell f$ is relatively weakly compact in $CB(G)$.  We denote the set
of all such functions by $WAP(G)$.  $WAP(G)$ is an
introverted $C^*$-subalgebra of $CB(G)$ containing the constant functions
(see~\cite[Theorems 4.2 and 12.1]{eberlein}).

A function $f\in CB(G)$ is called {\it left uniformly continuous}
when the map $a\mapsto \ell_a f$ from $G$ into $CB(G)$ is continuous.
We denote the set of all such functions by $LUC(G)$.
$LUC(G)$ is an invariant, left introverted $C^*$-subalgebra
of $CB(G)$ containing the constant functions (see~\cite[pp. 64, 68, 72]{namioka}).

It is well known that equivalently, a function $f\in CB(G)$ is in
$AP(G)$ [resp. $WAP(G)$] if the right orbit $O_r f$ is relatively [weakly]
compact in $CB(G)$. Furthermore, Grothendieck proved that $f\in WAP(G)$ is equivalent to
the equality
$\lim_i \lim_j f(a_i b_j) = \lim_j \lim_i f(a_i b_j)$, whenever the limits exist,
where $\{a_i\}$ and $\{b_j\}$ are sequences in $G$~\cite{groth}.

Since the norm topology of $CB(G)$ is stronger than the weak topology, we know that $AP(G) \subseteq WAP(G)$.
It can also be shown that $AP(G) \subseteq LUC(G)$.
In general, there is no inclusion relationship between $WAP(G)$ and $LUC(G)$
(see~\cite[p.165]{milnes} for an example).
However, when $G$ is a locally compact topological group, $WAP(G)\subseteq LUC(G)$,
$C_0(G)\cap AP(G) = \{0\}$ and $C_0(G)\subseteq WAP(G)$.  Consequently,
$WAP(G) = AP(G)$ if and only if $G$ is a compact topological group.
Thus, in the case that $G$ is a compact topological group, $C_0(G) = AP(G) = WAP(G) = LUC(G) = CB(G)$.

\section{Means and Amenability}
We state the definitions for a mean and invariant mean on subspaces of $\ell^\infty(G)$.
We state some well known results concerning means, and we state the definitions of amenability
for semigroups and for discrete groups.

Let $X$ be an arbitrary set, and let $Y$ be a norm closed subspace of $\ell^\infty(X)$ containing the constant functions.
Denote the characteristic function of a subset $A\subseteq X$ by $\chi_A$.
A {\it mean} on $Y$ is any linear functional $m\in Y^*$ such that $m(\chi_X) = 1 = \|m\|$.
We denote the set of all means on $Y$ by $M(Y)$.

\begin{remark}\label{dense}
(a) The set $M(Y)$ is a convex, weak* compact subset of $\ball(Y^*)$, the unit ball of $Y^*$.
(b) We define the evaluation map $\delta\colon X \rightarrow \ball(Y^*)$ by
\[
\delta(x) = \delta_x,
\]
where
\[
\delta_x (f) = f(x)
\]
for all $f\in Y$.  The set $\conv \{\delta_x: x\in X\}$ is weak* dense in $M(Y)$
(see~\cite[p. 281]{day:1950} and~\cite[p. 513]{day:first}), and the set
$\cconv\{\delta_x: x\in X\}$ is weak* dense in $\ol{\ball(Y^*)}^{w^*}$.
\end{remark}

Let $Y$ be a left [resp. right] invariant norm closed subspace of $\ell^\infty(G)$ containing the constant
functions.  A mean $m\in Y^*$ is said to be a {\it left} [resp {\it right}\,] {\it invariant mean}
(LIM [resp. RIM]) if $m(\ell_g f) = m(f)$ [resp. $m(r_g f) = m(f)$] for all $g\in G$
and $f\in Y$. We denote the set of all such means by $LIM(Y)$ [resp. $RIM(Y)$].
We say $m$ is an {\it invariant mean} (IM) if $m$ is both a LIM and a RIM

$G$ is said be to {\it left} [resp. {\it right}\,] {\it amenable} if there exists a LIM [resp. RIM] on $\ell^\infty(G)$.
$G$ is {\it amenable} if there exists an IM on $\ell^\infty(G)$.

Let $G$ be a topological group.
It is well known that there exists a LIM for $\ell^\infty(G_d)$. In this case, there is also
a RIM for $\ell^\infty(G_d)$.  For this reason, a topological group $G$ is said to be
{\it amenable as discrete} if there exists a LIM, RIM, or an IM for $\ell^\infty(G_d)$.

%Let $G$ be a locally compact group with fixed left Haar measure $\mu$.  The definitions for
%(left and right) invariant means for semigroups extend to topological groups by replacing the
%space $\ell^\infty(G)$ in the definitions above with $L^\infty(G)$, the space of all essentially bounded
%complex-valued functions on $G$.

%Proofs for the following two results are in Greenleaf's book~\cite{gl:inv}.
%The discrete version of Theorem~\ref{a:sgrp} is due to Day.

%\begin{theorem}[Greenleaf]
%Let $G$ be a locally compact group with fixed left Haar measure $\mu$.
%The following are equivalent.
%\begin{enumerate}[(i)]
%\item There is a LIM on $L^\infty(G)$
%\item There is a LIM on $CB(G)$
%\item There is a LIM on $LUC(G)$
%\end{enumerate}
%$G$ is called {\it amenable} if any (hence all) of the above hold.
%\end{theorem}

%\begin{theorem}\label{a:sgrp}
%If $G$ is an amenable group, then every closed subgroup of $G$ is also amenable.
%\end{theorem}

\section{Arens Product}
In this section we recall the Arens product on subspaces
of $\ell^\infty(G)$.  We refer the reader to \cite{arens} for a description of the construction
of the Arens multiplication, and to~\cite[\S 6]{day:first} for results on the application
of Arens' ideas to semigroup algebras.

Let $Y$ be a left introverted subspace of $\ell^\infty(G)$.
The {\it Arens product} is a map from $Y^* \times Y^*$ to $Y^*$ defined by:
\[
(m,n) \mapsto m\oodot n,
\]
where
\[
m\oodot n(f) = \langle m, n_\ell f\rangle
\]
for all $m$, $n\in Y^*$ and $f\in Y$.
The Arens product is associative, distributive, and weak*-weak*
continuous in the first variable.
Also,
\[
\|m\oodot n\| \leq \|m\|\|n\|,
\]
showing that the Arens product
makes $Y^*$ Banach algebra.  $\oodot$ is commonly referred to as the {\it first Arens product} on $Y^*$.
The {\it topological center} of $Y^*$ with respect to the first Arens product is defined to be
\[
Z_t(Y) = \{m\in Y^*:\text{ the map }n\mapsto m\oodot n\text{ is w*-w* continuous on }Y^*\}.
\]

When $Y$ is a right introverted subspace of $\ell^\infty(G)$,
the {\it second Arens product} is defined for $Y^*$ by
\[
(m,n) \mapsto m\,\Box\,n,
\]
where
\[
m\,\Box\,n(f) = \langle m, n_r f\rangle.
\]
The second Arens product is associative, distributive, and weak*-weak*
continuous in the first variable.  Furthermore, $Y^*$ is Banach algebra
with this product.

If $Y$ is a left [resp. right] introverted subspace of $\ell^\infty(G)$, then
$Y^*$ is a right topological semigroup under the first [resp. second] Arens product.

Let $Y$ be introverted.
It generally not the case that the first and second Arens products agree.
In this case, the {\it topological center} of $Y^*$ is defined to be
\[
Z_t(Y) = \{m\in Y^*:m\oodot n = m\,\Box\,n\text{ for all }n\in Y^*\}.
\]
$Y$ is called {\it Arens regular} when $Z_t(Y) = Y^*$,

The topological centers of various Banach algebras have been studied by
Dales and Lau~\cite{lau&dales}, Lau~\cite{lau:arens},
Lau and \"{U}lger~\cite{lau&ulger}, and Neufang~\cite{neufang}, to name a few.
For a discussion of the Arens products on the second
duals of Banach algebras, and for a discussion of the topological center of the Arens products in such a setting,
we refer the reader to~\cite{lau&ulger}.
The reader will also find many new results and examples regarding the topological center
of second duals of various Banach algebras in~\cite{lau&dales}.

\pagebreak
\section{Examples}

\begin{example}
Let $G$ be a topological group with a closed subgroup $H$.  $G$ has a jointly continuous left action
on the coset space $G/H$ defined by the map
\[
(a,gH)\mapsto agH.
\]
\end{example}

\begin{example}
Let $G = (\mathbb{N}, +)$, the commutative additive semigroup of the natural numbers.
Let $X = \beta\mathbb{N}$, the Stone-\u{C}ech compactification of $G$.
Addition in $G$ extends uniquely to a binary operation $\star$ on $X$ such that
$a+b = a\star b$ for every $a$, $b\in G$, and such that $G$ has a continuous 
left action on $X$ defined by
\[
(a, x) \mapsto a\star x,
\]
for $a\in G$ and $x\in X$.
\end{example}

\begin{example}[Bohr~\cite{bohr}]
Let $G=(\mathbb{R}, +)$, the additive semigroup of the real numbers, with the usual topology.
Every continuous periodic function on $G$ is almost periodic.  It follows that every trigonometric
polynomial
\[
f(x) = \sum_{n=1}^N \xi_n e^{i\alpha_n x},\qquad x,\alpha_n\in\mathbb{R},\,\xi_n\in \mathbb{C}
\]
is almost periodic.
\end{example}

\begin{example}
Let $G=(\mathbb{R}, +)$.  The function
\[
x\mapsto \frac{x}{1 + |x|}
\]
is a function in $LUC(G)$, but not in $WAP(G)$~\cite[4.4.19]{milnes}.
\end{example}

\begin{example}\label{free}
The free group on two generators is not amenable~\cite[17.16]{h&r}.
\end{example}

\begin{example}[Day]
As a discrete group, the special orthogonal group $SO(3,\mathbb{R})$ contains
the free group on two generators as a closed subgroup.  Day proved that
if a discrete group $G$ is amenable, then every closed subgroup $H$ of $G$ is
also amenable.
By Example~\ref{free} and the previous statement, $SO(3,\mathbb{R})$
is not amenable as a discrete group.
\end{example}

\begin{example}
The ``$ax+b$'' group, or affine group of $\mathbb{R}$, consists of matrices of the form
\[  \left[ \begin{array}{cc}
                  a & b \\
                  0 & 1
                  \end{array}
          \right],\qquad a\in\mathbb{R}^+,\,b\in\mathbb{R}
\]
with the topology inherited from the half plane in $\mathbb{R}^2$, and with the group operations of 
matrix inversion and matrix multiplication.
It is well known that the ``$ax+b$'' group is a solvable, hence amenable, locally compact group.
\end{example}

\begin{example}
The Heisenberg group over the real numbers consists of matrices of the form
\[ \left[ \begin{array}{ccc}
           1 & x & y \\
           0 & 1 & z \\
           0 & 0 & 1
          \end{array}
   \right],\qquad x,y,z\in\mathbb{R}
\]
with the topology inherited from $\mathbb{R}^3$, and with the group operations of 
matrix inversion and matrix multiplication.
The Heisenberg group is a nilpotent locally compact group.  Hence it is solvable, and therefore amenable.
\end{example}

\begin{example}
Let $\mathcal{H}$ be a Hilbert space, and denote the space of bounded linear operators from
$\mathcal{H}$ to $\mathcal{H}$ by $\mathcal{B}(\mathcal{H})$.
$\ball(\mathcal{B}(\mathcal{H}))$  with the weak operator topology (WOT) is
a compact, right topological semigroup, where the semigroup operation is the composition
of operators.

Let $S$, $T \in \ball(\mathcal{B}(\mathcal{H}))$.  Clearly $\|ST\| \leq \|S\|\|T\| \leq 1$.
To see that multiplication from the left is WOT continuous, take a net
$\{S_\alpha\} \subset \ball(\mathcal{B}(\mathcal{H}))$ such that $S_\alpha \rightarrow S$ in the WOT.
Then
\[
\langle (S_\alpha T)x, y\rangle = \langle S_\alpha(Tx), y\rangle \rightarrow \langle (ST)x, y\rangle
\]
for all $x$, $y\in\mathcal{H}$.  The WOT and weak* topologies agree on norm bounded sets,
hence $\ball(\mathcal{B}(\mathcal{H}))$ is WOT compact.
\end{example}