\chapter{Fixed Point Properties}

\section{Introduction}

This chapter is concerned with fixed point properties of locally compact groups
on compact convex subsets of locally convex topological spaces.
In Section~\ref{haar} we utilize Day's fixed point theorem in a functional
analysis proof of the existence
of a $G$-invariant measure on the coset space $G/H$, where $G$ is amenable as
a discrete group and $H$ is a closed subgroup of $G$.
Simon proved that a locally compact group $G$ has the strong fixed point
property if and only if it has the fixed point property~\cite{simon}.
In Section~\ref{simon} we prove a similar result in the setting of coset spaces.

\section{Existence of a $G$-invariant measure on coset spaces}\label{haar}

Let locally compact group $G$ be amenable as discrete, and let $H$ be a closed subgroup of $G$.
We prove, using Day's fixed point theorem~\cite{day:fpt},
that $G/H$ admits a $G$-invariant Radon measure; 
that is, a regular Borel measure $\mu$ on the coset space $G/H$ such that $\mu(gE) = 
\mu(E)$ for all $g \in G$ and all Borel sets $E$ of $G/H$.  The proof uses an idea
of Izzo in the proof of the existence of the Haar measure on locally compact
abelian groups using the Markov-Kakutani fixed point theorem (see~\cite{izzo:haar}).

Day's fixed point theorem is as follows.
\begin{theorem}[Day]
Let $K$ be a compact, convex subset of a locally convex Hausdorff 
topological vector space.  Let $S$ be a semigroup of affine continuous 
transformations of $K$ into itself.  If $S$ is amenable as a discrete 
semigroup, then there exists a point $k \in K$ such that $Tk=k$ for all $T 
\in S$.
\end{theorem}

For our proof, we also need the following two lemmas, both of which are
given in~\cite{izzo:haar}.  For the sake of completeness, we provide the
details of Izzo's argument.
\begin{lemma}\label{ginv:1}
Let $G$ be a topological group and let $U$ be a symmetric neighbourhood of 
the identity in $G$.  Then there exists a subset $V$ of $G$ such that for 
each $g \in G$, the set $gUU$ contains at least one element of 
$V$, and such that the set $gU$ contains at most one element of $V$.
\end{lemma}
\proof
Define the family ${\mathcal F}$ of subsets of $G$ by
\[
\mathcal{F} = \{ T \subset G: p^{-1}q \not\in UU\text{ for all }p,q\in T\}.
\]
${\mathcal F}$ has a maximal element $V$ by Zorn's Lemma.  Consider $g\in G$.
There exists some $v\in V$ such that $g^{-1}v\in UU$, for if this were not the case
then $V\cup\{g\} \in \mathcal{F}$, contradicting the maximality of $V$.  Thus
$gUU$ contains at least one element of $V$.

Now suppose that $gU$ contains two distinct elements $u,v\in V$.  Then
\[
u^{-1}v = u^{-1}gg^{-1}v \in U^{-1}U = UU,
\]
contradicting the definition
of $V$.  Thus $gU$ contains at most one element of $V$.
\done

\begin{lemma}\label{ginv:2}
Let $X$ be a vector space.
Give $X^*$ the weak* topology.  If $K$ is 
a closed subset of $X^*$ such that for each $x \in X$ the set 
$\{\phi(x):\phi\in K\}$ is bounded, then $K$ is compact.
\end{lemma}
\proof
For each $x\in X$, denote by $b(x)$ the least scalar such that $|\phi(x)| \leq b(x)$
for all $\phi\in K$, and let $D(x)$ be the set of all scalars $\alpha$ such that
$|\alpha| \leq b(x)$.  Let $I = \prod_{x\in X}D(x)$, and give it the product topology.
Tychonoff's theorem gives us that $I$ is compact with this topology.

The elements of $I$ are functions $\psi$
on $X$ such that $| \psi(x)| \leq b(x)$ for all $x\in X$.  Thus
$K\subset X^* \cap I$, and as such it has both the weak* and the product
topology of $X^*$ and $I$ respectively.  In fact, these topologies coincide on $K$.
Indeed, fix a $\phi_0 \in K$, and choose $\{x_i\}_{i=1}^n \subset X$ and $\epsilon > 0$.
We define the following sets,
\begin{align*}
W_1 &= \{\phi\in X^* : |\phi(x_i) - \phi_0(x_i)| < \epsilon,\;i=1,\dots,n\} \\
W_2 &= \{\psi\in I : |\psi(x_i) - \phi_0(x_i)| < \epsilon,\;i=1,\dots,n\}.
\end{align*}
Letting $\{x_i\}_{i=1}^n$ and $\epsilon$ range over all possible values, we have
that the sets $W_1$ form a local base of the weak* star topology of $X^*$ at $\phi_0$,
and also that the sets $W_2$ form a local base for the product topology of $I$
at $\phi_0$.  Now, since $K\subset X^*\cap I$, we have $W_1\cap K = W_2\cap K$.
Thus the two topologies coincide on $K$.

Next we claim that $K$ is closed in the product topology of $I$.  To prove this
claim, first consider some $\psi_0 \in \ol{K}^I$.  Suppose $x$, $y\in X$, $\alpha$, $\beta\in\mathbb{C}$,
and let $z=\alpha x+ \beta y$ and $\epsilon > 0$.  The set
\[
V_{z,\epsilon} = \{\psi\in I : | (\psi-\psi_0)(x)|,|(\psi-\psi_0)(y)|,|(\psi-\psi_0)(z)| < \epsilon\}
\]
is a product topology neighbourhood of $\psi_0$, and so there exists $\psi_{z,\epsilon}\in K\cap V_{z,\epsilon}$.
We have that $\psi_0$ is in $X^*$.  Indeed, for any $x$, $y\in X$, $\alpha$, $\beta\in\mathbb{C}$, any $\epsilon >0$,
and setting $z=\alpha x + \beta y$,
\begin{eqnarray*}
\lefteqn{
\psi_0(\alpha x + \beta y) - \alpha\psi_0 (x) - \beta\psi_0 (y) =} \\
 & & (\psi_0 - \psi_{z,\epsilon})(\alpha x + \beta y) + \alpha(\psi_{z,\epsilon} - \psi_0 )(x) + \beta(\psi_{z,\epsilon} - \psi_0 )(y),
\end{eqnarray*}
and thus
\[
| \psi_0(\alpha x + \beta y) - \alpha\psi_0 (x) - \beta\psi_0 (y)| < (1+| \alpha| + | \beta|)\epsilon.
\]
Finally, for $x\in X$ and $\epsilon >0$, a similar argument shows that there exists
a $\psi_{x,\epsilon} \in K$ such that
\[
|\psi_0(x)-\psi_{x,\epsilon}(x)| < \epsilon.
\]
Since $K$ is weak* closed, $\psi_0 \in K$.
$K$ is then compact in the product topology.  But
the topologies coincide on $K$, and thus $K$ is weak* compact in $X^*$.
\done

\begin{theorem}\label{coset}
Let $G$ be a locally compact group which is amenable as a discrete group,
and let $H$ be a closed subgroup of $G$.  Then $G/H$ admits a $G$-invariant
measure.
\end{theorem}
\proof
Let $G$ be a locally compact group which is amenable as discrete.  Let $H$ 
be a closed subgroup of $G$.
We proceed by showing that there is a nonzero positive linear functional 
on $C_c(G/H)$ which is invariant under the translations of $f \in 
C_c(G/H)$ by elements in $G$.
 
For each $a \in G$ we define $T_a: C_c(G/H)^*\to C_c(G/H)^*$ by
\[
\langle T_a,\phi\rangle(f) = \phi(\lambda_a f),\qquad \phi \in C_c(G/H)^*,\; f \in C_c(G/H).
\]
Each $T_a$ is continuous and affine, and $S = \{T_a : a\in G \}$ is a 
representation of $G$.

Fix a symmetric neighbourhood $U$ of the identity in $G$ such that the 
closure of $U$ is compact.  Let $K$ be all positive linear functionals 
$\phi \in C_c(G/H)^*$ which satisfy:
\begin{enumerate}
\item $\phi(f) \leq 1$ for all nonnegative $f \in C_c(G/H)$ that are 
bounded 
above by 1 and supported in $(xU)H = \{(xu)H: u\in U\}$ for some $x \in G$, and
\item $\phi(f) \geq 1$ for all nonnegative $f \in C_c(G/H)$ that are equal 
to 1 on $(xUU)H = \{(xuv)H: u,v\in U\}$ for some $x \in G$.
\end{enumerate}
Clearly $K$ is weak* closed and convex in $C_c(G/H)$.  Also notice that by 
a partition of unity argument, every nonnegative function in $C_c(G/H)$ 
can be expressed as a finite sum of nonnegative functions each with 
support contained in $(xU)H$ for some $x \in G/H$.  By condition 1 of the 
definition of $K$, it follows that for each $f \in C_c(G/H)$, the set 
$\{\phi(f):\phi\in K\}$ is bounded.  By Lemma~\ref{ginv:2}, $K$ 
is weak* compact.

To see that $K$ is nonempty, take $V$ as in Lemma~\ref{ginv:1} and consider
the functional
\[
\psi : f \mapsto \sum_{v\in V} f(vH).
\]
We show that $\psi \in K$.
Indeed, take $f_1$ as in condition 1 of the definition of $K$.  The support of $f_1$
is contained in $(aU)H$ for some $a\in G$.  $aU$ contains at most one element of $V$,
and thus
\[
\psi(f_1) = \sum_{v\in V} f_1(vH) \leq 1.
\]
Taking $f_2$ as in
condition 2 of the definition of $K$, we similarly notice that
\[
\psi(f_2) = \sum_{v\in V} f_2(vH) \geq 1.
\]
Thus $\psi\in K$.

It remains to show that each $T_a$ maps $K$ onto itself.  This follows 
from the definition of K; for if $\phi \in K$ and $f$ is as in condition 1 
of the definition of $K$, $\|\lambda_a f\| \leq 1$ and the support of $\lambda_a f$ is 
contained in $(a^{-1}gU)H$ for some $g \in G$.  Thus $\langle T_a,\phi\rangle(f) = 
\phi(\lambda_a f) \leq 1$.  Similarly, if $f$ is as in condition 2 of the 
definition of $K$, $\langle T_a,\phi\rangle(f) \geq 1$.

By Day's fixed point theorem, $S = \{T_a: a\in G\}$ has a common 
fixed point in $K$.  $K$ contains positive linear functionals on $C_c(G/H)$.
The Riesz representation theorem states that each
positive linear functional on $C_c(G/H)$ determines a regular Borel measure on $G/H$.
We have shown that there exists a regular Borel measure for $G/H$ which
is $G$ invariant.
\done

\begin{remark}
(a) Lemma~\ref{ginv:2} is essentially Alaoglu's theorem~\cite[V.4.2]{d&s}.
(b) The proof presented in Theorem~\ref{coset} is new even when $H = \{e\}$.
\end{remark}

\section{A Fixed Point Property for the pair\\$(G:H)$}\label{simon}
Let $G$ be a locally compact group with a closed subgroup $H$.
The homogeneous space $G/H$ possesses a quasi-invariant measure.  We denote the space
of all essentially bounded complex-valued functions on $G/H$ by $L^\infty(G/H)$.

Eymard defined the following fixed point property for the pair $(G:H)$ (see~\cite[p. 11]{eymard}).
\begin{defn}[Eymard]\label{fpp}
The pair $(G:H)$ is said to have the {\it fixed point property} (FPP) if every jointly continuous
affine action of $G$ on a compact convex subset $K$ of a locally convex
topological space $X$ which has a fixed point for $H$ also has a fixed point for $G$.
\end{defn}
Furthermore, Eymard proved that there exists a GLIM on $L^\infty(G/H)$ if and only if $(G:H)$
has the FPP.

In~\cite{simon}, Simon considered the following weaker condition of the action of $G$ on $K$.
\begin{defn}[Simon]
A {\it weakly measurable affine action} of $G$ on a compact convex subset $K$ of a locally
convex topological space $X$ is a representation of $G$ by continuous affine maps
$\alpha_g : K \rightarrow K$ such that for each $\phi \in X^*$ and $x\in K$, the map
$g\mapsto \langle\phi, \alpha_g (x)\rangle$ is measurable.
\end{defn}

We define a second fixed point property for the pair $(G:H)$, and we show
that $(G:H)$ possesses this fixed point property if and only if it has the FPP.
\begin{defn}
The pair $(G:H)$ is said to have the
%the {\it strong fixed point property}
FPP2 if every weakly
measurable affine action of $G$ on a compact convex subset $K$ of a locally convex
topological space $X$ which has a fixed point for $H$ also has a fixed point for $G$.
\end{defn}
\pagebreak
\begin{theorem}
For $G$ a locally compact group with a closed subgroup $H$, the following are equivalent:
\begin{enumerate}[(i)]
\item There exists a GLIM on $L^\infty (G/H)$.
\item $(G:H)$ has the FPP2.
\item $(G:H)$ has the FPP.
\item There exists a GLIM on $LUC(G,G/H)$.
\end{enumerate}
\end{theorem}

\proof
(ii) $\Rightarrow$ (iii) is trivial, since every jointly continuous affine action is
also a weakly measurable affine action.

(iii) $\Leftrightarrow$ (iv) follows immediately from the fixed point theorem of Eymard~\cite[p. 12]{eymard},
and a theorem of Greenleaf~\cite[Theorem 3.3]{gl:aa}.

That (iv) $\Leftrightarrow$ (i) is due to Greenleaf \cite[Theorem 3.3]{gl:aa}.

(i) $\Rightarrow$ (ii)  Consider a weakly measurable action of $G$ on $K$ such that
\[
hx_o = \alpha_h (x_o) = x_o
\]
 for all $h\in H$.
Let $\mathcal{A}(K)$ be the Banach space of all continuous affine maps from $K$ to $K$.
For every $\phi \in \mathcal{A}(K)$, we define the function $f_\phi$ by
\[
f_\phi (gH) = \langle\phi, g x_o\rangle.
\]
$\phi$ is bounded on $K$.  Thus $f_\phi \in L^\infty (G/H)$ since the action
is weakly measurable.
Let $m$ be the left invariant mean on $L^\infty (G/H)$.

Next, we define a function $T$ on $\mathcal{A}(K)$ by
\[
\langle T, \phi \rangle = \langle m, f_\phi \rangle.
\]
Notice that $T$ is linear.  Indeed, let $\alpha$, $\beta \in {\mathbb C}$, and
let $\phi$, $\psi \in \mathcal{A}(K)$. Then,
\begin{align*}
f_{\alpha\phi+\beta\psi}(gH) &= \langle \alpha\phi+\beta\psi,g x_o \rangle
= \alpha\langle\phi , g x_o \rangle + \beta\langle\psi, g x_o \rangle \\
&= (\alpha f_\phi + \beta f_\psi)(gH).
\end{align*}
This gives us that
\[
\langle T, \alpha\phi + \beta\psi \rangle = \langle m, f_{\alpha\phi + \beta\psi} \rangle
= \langle m, \alpha f_\phi + \beta f_\psi \rangle
= \alpha\langle T, \phi \rangle + \beta\langle T, \psi \rangle.
\]
Moreover, from the definition of $T$ we find that
$\inf_{k\in K} \phi(k) \leq \langle T, \phi \rangle \leq \sup_{k\in K} \phi(k)$ for any real
valued $\phi$.  Thus $T|_{X^*} \in X^{**}$.

Let $Q$ be the canonical embedding of $X$ into $X^{**}$ via $\langle Q(x),\phi\rangle = \phi(x)$.
We proceed by first showing that for some $k_o\in X$, $T|_{X^*} = Q(k_o)$.  Then we
use the existence of a left invariant mean $m$ to show that $k_o$ is a fixed point for $G$.

To show that $T|_{X^*} = Q(k_o)$ for some $k_o\in X$, we must show that $T$ is
weak* continuous.
The Mackey topology on $X^*$ is the finest topology such that $X^*$ has $X^{**}$ as its dual
space.  Thus, it suffices to show that the convergence of $\phi_\gamma$ to $\phi$ in the Mackey
topology implies convergence of $\langle T, \phi_\gamma \rangle$ to $\langle T, \phi \rangle$.
$\phi_\gamma$ converges to $\phi$ in the Mackey topology if $\phi_\gamma$ converges 
uniformly to $\phi$ on convex weakly compact subsets of $X^{**}$.  But $Q(K)$ is such a set.  In
particular, we may assume that
\[
\langle \phi_\gamma ,g x_o \rangle = f_{\phi_\gamma}(gH)
\rightarrow f_{\phi}(gH) = \langle \phi, g x_o \rangle
\]
uniformly, i.e., $f_{\phi_\gamma}$ is norm
convergent to $f_\phi$ in $L^\infty (G/H)$.  Now $m$ is a mean on
$L^\infty (G/H)$, and so $T$ is indeed weak* continuous.

For any $\phi\in X^*$ and any $a, g\in G$, notice that
\begin{equation}\label{fp1}
f_{\lambda_a \phi}(gH) = (\lambda_a \phi)(g x_o) =  \lambda_a(\phi(g x_o)) =  \lambda_a(f_\phi)(gH).
\end{equation}
Then, for all $g \in G$,
\begin{align*}
\phi(g k_o) &= (\lambda_g \phi)(k_o) = \langle T, \lambda_g \phi \rangle = \langle m, f_{\lambda_g \phi} \rangle
\stackrel{(\ref{fp1})}{=} \langle m, \lambda_g (f_\phi) \rangle \\
&= \langle m, f_\phi \rangle = \langle T, \phi \rangle = \phi(k_o),
\end{align*}
by the invariance of $m$.
Hence $g k_o = k_o$ for all $g\in G$.
\done
