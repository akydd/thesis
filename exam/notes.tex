\documentclass[12pt]{report}
\setlength{\textheight}{8.5in}
\setlength{\textwidth}{5.5in} 
\setlength{\topmargin}{-1.3cm}
\renewcommand{\baselinestretch}{1.37}

\usepackage{mathesis, def, enumerate, amsmath, amsfonts, amssymb}
\usepackage{wasysym}

% Try kdvi inverse search feature!
\usepackage[active]{srcltx}

\begin{document}
\chapter{Examples}
\section{Group Theory}
Let $G$ be a group.

\begin{defn}
Let $a\in G$.  The {\it cyclic group (of $G$) generated by $a$}, which we denote by
$\langle a\rangle$, is the set of all powers of $a$.
\end{defn}

\begin{defn}
If $a$, $b\in G$, the {\it commutator} of $a$ and $b$, denoted $[a,b]$, is
\[
aba^{-1}b^{-1}.
\]
The {\it commutator subgroup} of $G$, denoted $G^{(1)}$, is the subgroup of $G$
generated by all the commutators.  $G^{(1)}$ is a normal subgroup of $G$
($G^{(1)} \lhd G$).
\end{defn}

\begin{defn}
If $H$ and $K$ are both subgroups of $G$, then
\[
[H,K] = \langle [h,k]:h\in H,\:k\in K\rangle.
\]
Notice that $[G,G] = G^{(1)}$, and, more generally, $G^{(n+1)} = [G^{(n)}, G^{(n)}]$.
The series $G = G^{(0)} \geq G^{(1)} \geq G^{(2)} \geq\cdots$ is called the {\it derived series}
of $G$.
\end{defn}

\begin{defn}
The {\it characteristic subgroups $\gamma_i(G)$} of $G$ are defined inductively:
\[
\gamma_1(G) = G,\qquad\gamma_{i+1}(G) = [\gamma_i(G),G].
\]
\end{defn}

\begin{defn}
$G$ is {\it nilpotent} if $\gamma_i(G) = \{e\}$ for some positive integer $i$.
The least such $i$ is called the {\it class} of the nilpotent group.
\end{defn}

\begin{defn}
$G$ is {\it solvable} if there exists a finite series of subgroups
\[
\{e\} = G_n \lhd G_{n-1} \lhd \dots \lhd G_1 \lhd G_0 = G
\]
such that each factor group $G_i/G_{i+1}$ is abelian ($i=0,\dots ,n-1$).
In this case, such a series is called a {\it solvable series}.
\end{defn}

\begin{lemma}
If $\{e\} = G_n \lhd G_{n-1} \lhd \dots \lhd G_1 \lhd G_0 = G$ is a solvable series,
then $G^{(i)} \leq G_i$ for all $i$.
\end{lemma}
%\proof
%\done

\begin{theorem}
$G$ is solvable if and only if $G^{(n)} = \{e\}$ for some positive $n$.
\end{theorem}
\proof
`` $\Rightarrow$ '' follows from the above lemma.

Conversely, if $G^{(n)} = \{e\}$, it is a fact that the derived series is a normal series.
\done

\begin{proposition}
Every nilpotent group is solvable.
\end{proposition}
\proof
By induction, $G^{(i)} \leq \gamma_i (G)$.  Indeed, $G^{(1)} = [G,G] = [\gamma_1(G), G]
= \gamma_2(G)$.  Assume now that $G^{(i-1)} \leq \gamma_{i-1}(G)$.  Then,
\begin{align*}
G^{(i)} &= \langle[g,h]: g,h\in G^{(i-1)} \rangle\\
\gamma_i (G) &= \langle[x,y]:x\in \gamma_{i-1}(G),y\in G\rangle.
\end{align*}
An element of $G^{(i)}$ looks like $(ghg^{-1}h^{-1})^k$.  But $h,h^{-1}\in G$.
By assumption, $g,g^{-1}\in \gamma_{i-1}(G)$.  Thus $(ghg^{-1}h^{-1})^k \in \gamma_i(G)$.
Next, if $\gamma_{c+1}(G) = \{e\}$, then $G^{(c+1)} = \{e\}$  Thus $G$ is solvable, with
derived length of $c+1$.
\done

\begin{example}
The affine group is solvable.
\end{example}
\proof
First calculate $G^{(1)} = [G,G]$.
\[
\begin{bmatrix}a & b \\ 0 & 1 \end{bmatrix}
\begin{bmatrix}c & d \\ 0 & 1 \end{bmatrix}
\left( \begin{bmatrix}c & d \\ 0 & 1 \end{bmatrix} \begin{bmatrix}a & b \\ 0 & 1 \end{bmatrix} \right)^{-1}
=
\begin{bmatrix} 1 & d(a-1) + b(1-c) \\ 0 & 1\end{bmatrix}.
\]
Thus
\[
G^{(1)} = \left<\begin{bmatrix} 1 & x\\0 & 1\end{bmatrix}:x\in\mathbb{R}\right>.
\]
Next calculate $G^{(2)} = [G^{(1)},G^{(1)}]$.
Notice that for any $x,y\in \mathbb{R}$,
\[
\begin{bmatrix}1 & x \\ 0 & 1 \end{bmatrix}
\begin{bmatrix}1 & y \\ 0 & 1 \end{bmatrix}
\left(\begin{bmatrix}1 & y \\ 0 & 1 \end{bmatrix} \begin{bmatrix}1 & x \\ 0 & 1 \end{bmatrix}\right)^{-1}
= \begin{bmatrix} 1 & 0\\0 & 1\end{bmatrix}.
\]
Thus $G^{(2)} = \{e\}$.
\done

\begin{example}
The Heisenberg group is nilpotent.
\end{example}
\proof
$\gamma_1(G) = G$.
Calculate $\gamma_2(G) = [\gamma_1(G), G] = [G,G] = G^{(1)}$:  Notice that
\[
\begin{bmatrix}1 & a & b\\0 & 1 & c\\0 & 0 & 1\end{bmatrix}
\begin{bmatrix}1 & d & e\\0 & 1 & f\\0 & 0 & 1\end{bmatrix}
\left(\begin{bmatrix}1 & d & e\\0 & 1 & f\\0 & 0 & 1\end{bmatrix}\begin{bmatrix}1 & a & b\\0 & 1 & c\\0 & 0 & 1\end{bmatrix}\right)^{-1}
=\begin{bmatrix}1 & 0 & af-cd\\0 & 1 & 0\\0 & 0 & 1\end{bmatrix}.
\]
Thus
\[
\gamma_2(G) = \left<\begin{bmatrix}1 & 0 & x\\0 & 1 & 0\\0 & 0 & 1\end{bmatrix}: x\in\mathbb{R}\right>.
\]
Calculate $\gamma_3(G) = [\gamma_2(G),G]$:
\[
\begin{bmatrix}1 & 0 & x\\0 & 1 & 0\\0 & 0 & 1\end{bmatrix}
\begin{bmatrix}1 & a & b\\0 & 1 & c\\0 & 0 & 1\end{bmatrix}
\left(\begin{bmatrix}1 & a & b\\0 & 1 & c\\0 & 0 & 1\end{bmatrix}\begin{bmatrix}1 & 0 & x\\0 & 1 & 0\\0 & 0 & 1\end{bmatrix}\right)^{-1}
=\begin{bmatrix}1 & 0 & 0\\0 & 1 & 0\\0 & 0 & 1\end{bmatrix}.
\]
Thus $\gamma_3(G) = \{e\}$.
\done

\section{Amenability}
\begin{defn}
Let $G$ be a l.c. group.  A {\it mean} on $L^\infty(G)$ is a functional $m\in L^\infty(G)^*$
such that $m(1) = \|m\| = 1$.
\end{defn}

\begin{theorem}
\begin{enumerate}[(i)]
\item $m\in L^\infty(G)^*$ is a mean iff $m(1) = 1$ and $m(f) \geq 0$ whenever $f\geq 0$.
\item If $m$ is a mean, then
\[
\inf_{x\in G}f(x) \leq m(f) \leq \sup_{x\in G}f(x)
\]
for all $\mathbb{R}$-valued $f\in L^\infty(G)$.
\end{enumerate}
\end{theorem}
\proof
(i) `` $\Rightarrow$ ''  $m(1)=1$ by definition.  Let $f\geq 0$ with $\|f\| \leq 1$ (wlog).
Then $\|1-f\| = \sup_{x\in G}|1-f(x)| \leq 1$.  Thus
\[
m(1) - m(f) = m(1-f) \leq |m(1-f)| = \|m\|\|1-f\| \leq 1,
\]
and so $m(f) \geq 0$.

`` $\Leftarrow$ ''
Consider (nonzero) real-valued $f$.  $\frac{f}{\|f\|} \leq 1$, and so $1-\frac{f}{\|f\|} \geq 0$.
By assumption, $m\left(1-\frac{f}{\|f\|}\right) \geq 0$, so
\[
\|f\|m(1) \geq m(f).
\]
For an arbitrary $\mathbb{C}$-valued $f$, choose $c\in\mathbb{C}$ such that $|c|$ and
$|m(f)| = cm(f)$.  Let $g=\Re(cf)$ and $h=\Im(cf)$.  Then
\begin{align*}
|m(f)| &= m(cf) = m(g) + im(h) = m(g)\quad\text{ (since $m(h)$ is $\mathbb{R}$-valued)}\\
&\leq \|g\|m(1) \leq \|cf\|m(1) = \|f\|m(1).
\end{align*}
Thus $\|m\| = m(1) = 1$.

(ii)
The functions
\[
f_1(x) = \sup f - f(x),\qquad f_2(x) = f(x) - \inf f
\]
are positive.
\done
\begin{theorem}
Let $G$ be a group.  If $G_d$ has a LIM, it has a RIM (and an IM)
\end{theorem}
\proof
For all $f\in \ell^\infty(G_d)$, define $f^*(x) = f(x^{-1})$.  Notice that for $a,g\in G$,
\[
(r_a f)^*(g) = r_a f(g^{-1}) = f(g^{-1}a) = f^*(a^{-1}g) = \ell_{a^{-1}}(f^*)(g).
\]
If $m\in LIM(\ell^\infty(G))$, then define $n(f) = m(f^*)$.
We have that
\[
n(r_a f) = m((r_a f)^*) = m(\ell_{a^{-1}}(f)^*) = m(f^*) = n(f).
\]
Thus $n\in RIM(\ell^\infty(G_d))$.
\done

\begin{theorem}\label{abelian}
Abelian groups are amenable.
\end{theorem}
\proof
Let $G$ be an abelian group.
Let $\{f_1,\dots ,f_n\} \subset \ell^\infty(G_d)$, and let $\{g_1,\dots ,g_n\} \subset G$.
Let
\[
h = \sum_{k=1}^n \left( f_k - \ell_{g_k}f_k\right).
\]
Let $\epsilon > 0$, and suppose, for a contradiction, that
\[
\sup_{x\in G} h(x) = -\epsilon.\tag{1}
\]
Let $p\in\mathbb{Z}^+$, and let $\Phi = \{f:\{1,\dots ,n\}\rightarrow \{1,\dots ,p\}\}$.
Clearly $| \Phi | = p^n$.  Define another function $\tau:\Phi\rightarrow G$ by
\[
\tau(\phi) = \prod_{k=1}^n g_k^{\phi(k)}.
\]
Fix $k$.  Then
\[
\sum_{\phi\in\Phi} \left(f_k(\tau(\phi)) - f_k(g_k\tau(\phi))\right) = 
\sum_{\phi\in\Phi} \left(f_k(g_1^{\phi(k)}\cdots g_n^{\phi(k)}) - f_k(g_1^{\phi(k)}\cdots g_k^{\phi(k)+1}\cdots g_n^{\phi(k)})\right).
\]
All terms above cancel, except for those $f_k(\tau(\phi))$ such that $\phi(k) = 1$ and those $f_k(g_k\tau(\phi))$
such that $\phi(k) = p$.  This is because the range of $\phi$ is in $\{1,\dots ,p\}$.  There are $p^{n-1}$ of each
of these elements, for a total of $2p^{n-1}$.  Thus
\[
\sum_{\phi\in\Phi} \left(f_k(\tau(\phi)) - f_k(g_k\tau(\phi))\right) \geq -2p^{n-1}\|f_k\|.
\]
By (1),
\begin{align*}
-\epsilon p^n \geq \sum_{\phi\in\Phi}h(\tau(\phi))=& \sum_{\phi\in\Phi}\sum_{k=1}^n (f_k(\tau(\phi)) - f_k(g_k\tau(\phi))) \\
=& \sum_{k=1}^n\sum_{\phi\in\Phi}(f_k(\tau(\phi)) - f_k(g_k\tau(\phi))) \\
\leq& -\sum_{k=1}^n 2p^{n-1}\max\{\|f_k\|\} \\
=& -2np^{n-1}\max\{\|f_k\|\}.
\end{align*}
Thus we get that
\[
\epsilon p \leq 2n\max\{\|f_k\|\},
\]
and taking the limit $p\rightarrow\infty$ we arrive at a contradiction: $0<\epsilon\leq 0$.
\done

A second proof used the Markov-Kakutani fixed point theorem:
\begin{theorem}[Markov-Kakutani Fixed Point Theorem]
Let $K$ be a compact, convex subset of a locally convex linear topological space $X$.
Let $S$ be a commutative family of continuous affine maps from $K$ to $K$.  Then
there exists a point $k\in K$ such that $T(k) = k$ for all $T\in S$.
\end{theorem}

\proof (of the amenability of an Abelian group):
The set $K$ of all means is a w*-compact, convex subset of $L^\infty(G)^*$.
Define a family of maps from $K$ to $K$ by
\[
\langle T_a(m),f\rangle = \langle m, \ell_a f\rangle,
\]
for $a\in G$, $f\in L^\infty(G)$.  The family $\{T_a:a\in G\}$ is commuting
since $G$ is Abelian.  Thus there is an $m\in K$ such that $m(f) = m(\ell_a f)$
for all $a\in G$ and $f\in L^\infty(G)$.
\done


\begin{theorem}\label{extend}
If $H\lhd G$ and if $H$ and $G/H$ are amenable, then $G$ is amenable.
\end{theorem}
\proof
Let $m\in LIM(\ell^\infty(H))$ and $n\in LIM(\ell^\infty(G/H))$.
For $f\in \ell^\infty(G)$ and $g\in G$, let $\hat{f}(g) = m(\ell_g f|_H)$.
We claim that the function $\hat{f}$ is constant on cosets.  i.e.,
suppose $xH = yH$.
Then, $x = yh_0$ for some $h_0\in H$.  For $h\in H$,
\[
\ell_x f(h) = f(xh) = f(yh_0 h) = \ell_{h_0}(\ell_y f)(h).
\]
Thus
\[
\hat{f}(x) = m(\ell_x f|_H) = m(\ell_{h_0}(\ell_y f)|_H) = m(\ell_y f|_H) = \hat{f}(y).
\]
Since $\hat{f}$ is constant on cosets, we may define $\tilde{f}\in\ell^\infty(G/H)$ via
\[
\tilde{f}(xH) = \hat{f}(x).
\]
The we define a mean $\phi(f) = n(\tilde{f})$.  To see that $\phi$ is left invariant, for $a,g\in G$,
\[
\widetilde{\ell_a f}(xH) = \hat{f}(ax) = \tilde{f}(axH) = \ell_{aH}\tilde{f}(xH),
\]
and thus,
\[
\phi(\ell_a f) = n(\widetilde{\ell_a f}) = n(\ell_{aH}\tilde{f}) = n(\tilde{f}) = \phi(f).
\]
\done

\begin{proposition}
A solvable group is amenable.
\end{proposition}
\proof
This follows from repeated applications of Theorems~\ref{abelian} and~\ref{extend}.
\done

\begin{example}
The free group on two generators is not amenable.
\end{example}
\proof
A reduced word $x$ contains no subwords of the form $aa^{-1}$, $a^{-1}a$, $bb^{-1}$ or $b^{-1}b$.
Divide $G$ into the disjoint sets $\{H_i:i\in\mathbb{Z}\}$ where
\[
x\in H_i \Leftrightarrow x=a^{i}b^{n_1}a^{n_2}\cdots, \text{ as a reduced word},
\]
where $\{n_j\}$ is any sequence in $\mathbb{Z}$ with $n_1\neq 0$, unless $x=a^i$.  Clearly
$G=\cup_{i\in\mathbb{Z}} H_i$.

Notice that $\ell_a (H_i) = H_{i+1}$.  Also, $\ell_a \chi_{H_i} = \chi_{aH_i} = \chi_{H_{i-1}}$.
For any mean $m$ and a fixed $k\in\mathbb{Z}$,
\[
1=m(\chi_G) = \sum_{i<k}m(\chi_{H_i}) + \sum_{i\geq k}m(\chi_{H_i}).
\]
If $m$ is left invariant, then
\begin{align*}
1=m(\ell_a \chi_G) &= \sum_{i<k}m(\chi_{H_{i-1}}) + \sum_{i\geq k}m(\chi_{H_i}) \\
&= \sum_{j\leq k}m(\chi_{H_{j}}) + \sum_{i\geq k}m(\chi_{H_i}),
\end{align*}
where we make the substitution $j=i-1$.  Thus $m(\chi_{H_k}) = 0$.  Since $k$ was arbitrary, it
follows that $m(\chi_{H_i})=0$ for all $i\in\mathbb{Z}$.

Next notice that $\ell_b (H_i) \subsetneq H_0$.  Assuming still that $m$ is a LIM,
\[
\sum_{i\neq 0} m(\chi_{H_i}) = m(\chi_{\cup_{i\neq 0} H_i}) = m(\ell_b \chi_{\cup_{i\neq 0} H_i})
= m(\chi_{\cup_{i\neq 0} bH_i}) \leq m(\chi_{H_0}).
\]
But since
\[
1=m(\chi_G) = m(\chi_{H_0}) + \sum_{i\neq 0}m(\chi_{H_i}),
\]
we have that
\[
m(\chi_{H_0}) \geq \frac{1}{2},
\]
a contradiction.
\done

\section{Functions on groups and semigroups}

\begin{example}
Let $G = (\mathbb{R}, +),$ the addivitive semigroup of reals.  The function
\[
f(x) = \sum_{k=1}^N \xi_k e^{ia_k x}
\]
is almost periodic.
\end{example}
\proof
For this $G$, almost periodicity is equivalent to the following definition.
\begin{defn}
$f\in CB(\mathbb{R})$ is almost periodic iff for any $\epsilon >0$ there exists a positive constant
$d_\epsilon$ such that every interval of length $d_\epsilon$ contains a number $t$ with the property that
$\|\ell_t f - f\| < \epsilon$
\end{defn}
The function on $x\mapsto e^{ix}$ has period $2\pi$. Let $\epsilon > 0$.  Take $d_\epsilon = 2\pi$.
Clearly this function is in $AP(G)$.  Since $AP(G)$ is an algebra, the result follows.
\done

\begin{example}
Let $G = (\mathbb{R}, +),$ the addivitive semigroup of reals.  The function
\[
f(x) = \frac{x}{1+|x|}
\]
is in $LUC(G)$ but not it $WAP(G)$.
\end{example}
\proof
To see that $f\in LUC(G)$, let $y_\alpha \rightarrow y \in G$.  Then
\begin{align*}
\|\ell_{y_\alpha} f - \ell_y f\| =& \sup_{x\in G}\left| \frac{y_\alpha +x}{1+|y_\alpha +x|} - \frac{y+x}{1+|y+x|} \right|\\
=&\sup_{x\in G}\left| \frac{y_\alpha + y_\alpha |y+x| + x + x|y+x| - y - y|y_\alpha + x| - x - x|y_\alpha + x|}{1+|y+x|+|y_\alpha + x| + |y_\alpha + x| |y+x|} \right|.
\end{align*}
Consider the numerator:
\begin{align*}
& |y_\alpha - y + y_\alpha|y+x| - y|y_\alpha +x| + x-x + x|y+x| - x|y_\alpha +x| | \\
\leq & |y_\alpha - y| + |y_\alpha|y+x| - y|y_\alpha+x|| + |x(|y+x| - |y_\alpha +x|)| \\
\leq & |y_\alpha - y| + |y_\alpha|y+x| - y|y+x| + y|y+x| - y|y_\alpha +x|| + |x|||y+x| - |y_\alpha +x|| \\
\leq & |y_\alpha - y| + |y+x||y_\alpha -y| + |y(|y+x| - |y_\alpha +x|)| + |x||y-y_\alpha| \\
\leq & |y_\alpha - y| + |y+x||y_\alpha -y| + |y||y-y_\alpha| + |x||y-y_\alpha|\rightarrow 0.
\end{align*}
Thus $f\in LUC(G)$.
\done

\chapter{Other Theorems}
\section{Day's Fixed Point Theorem}

\begin{theorem}[Day's Fixed Point Theorem]
Let $K$ be a convex, compact subset of a locally convex linear topological space $X$.
Let $S$ be a semigroup (under composition) of continuous affine maps from $K$ to $K$.
If $S$ is amenable as discrete, then there exists $k\in K$ such that $T(k) = k$
for all $T\in S$.
\end{theorem}
\proof (Outline)
Let $y\in K$.  Define linear map $F:X^*\rightarrow \ell^\infty(S)$ by
\[
\langle F\phi,T\rangle = \langle \phi, T(y)\rangle.
\]
$F$ is linear.  $\phi$ is bounded on the compact set $K$, thus $F(\phi)\in \ell^\infty(S)$.
Then we have the adjoint map, $F^\#: \ell^\infty(S)^*\rightarrow X^{*\#}$,
\[
\langle F^\# \mu,\phi\rangle = \langle \mu, F\phi\rangle.
\]

Let $K' = Q(K)$, where $Q:X\rightarrow X^{*\#}$, $Q(x)(\phi) = \phi(x)$.  It can be shown
that $F^\# \mu \in K'\subset X^{*\#}$ for all means $\mu\in\ell^\infty(S)^*$.
But then $Q^{-1}(F^\#\mu) \in K$.  i.e. the map $Q^{-1}F^\#$ maps all means into $K$.

Notice that if $\mu$ is left invariant, then
\[
\langle \ell_T^* \mu, f\rangle = \langle \mu, \ell_T f\rangle = \langle\mu,f\rangle
\]
for all $f\in \ell^\infty(S)$ and $T\in S$.
It can be shown that for any mean $\mu$ and $T\in S$ that
\[
Q^{-1}F^\#(\ell_T^*\mu) = T(Q^{-1}F^\#(\mu)).
\]
Thus, for left invariant $\mu$,
\[
T(Q^{-1}F^\#(\mu)) = Q^{-1}F^\#(\ell_T^*\mu) = Q^{-1}F^\#(\mu).
\]
i.e., the point $Q^{-1}F^\#(\mu)$ is a fixed point in $K$.
\done

\chapter{More definitions}

\section{What is $\beta\mathbb{N}$?}

\begin{defn}
A topological space $X$ is {\it regular} if for every closed set $C\subset X$ and point $x\not\in C$,
there exists two open disjoint sets $U$ and $V$ such that $C\subset U$ and $x\in V$.
$X$ is {\it completely regular} if there exists a continuous function $f:X\rightarrow [0,1]$ such that
$f(x) = 0$ and $f(C) = 1$.
\end{defn}

\begin{example}
A discrete space is completely regular.
\end{example}

\begin{defn}
Let $X$ be a topological space.  The {\it Stone-\u{C}ech compactification} of
$X$ is a compact space, $\beta X$, with an embedding $\kappa:X\rightarrow\beta X$
such that $\kappa(X)$ is dense in $\beta X$, and such that every continuous
function $f$ from $X$ to any compact Hausdorff space $K$ extends to a
continuous $\tilde{f}:\beta X\rightarrow K$.  $\beta X$ is unique up to homeomorphism (bijection, continuous
and inversely continuous).
\end{defn}

\begin{example}
Let $X=\mathbb{N}$.  The space of all multiplicative means for $\mathbb{N}$, $MM(\ell^\infty(\mathbb{N}))$
is weak* compact.  The mapping $x\mapsto \delta_x$ maps $\mathbb{N}$ into $MM(\ell^\infty(\mathbb{N}))$ such
that $\{\delta_x : x\in\mathbb{N}\}$ is weak* dense in $MM(\ell^\infty(\mathbb{N}))$.  This is just the
character space of the commutative $C^*$-algebra $\ell^\infty(\mathbb{N})$.
\end{example}

\section{Quasi Invariant Measures}

\begin{defn}
Let $\lambda$ be a Radon measure on the homogenous space $G/H$.
Define
\[
\lambda_x(E) = \lambda(xE)
\]
for $x\in G$.  $\lambda$ is {\it quasi-invariant} if there
exists a continuous function $f:G\times (G/H) \rightarrow (0,\infty)$ such that
\[
d\lambda_x(p) = f(x,p)d\lambda(p)
\]
for all $x\in G$ and $p\in G/H$
\end{defn}

\begin{proposition}
For any lcg $G$ and closed subgroup $H$, $(G:H)$ admits a function
\[
\rho(x\xi) = \frac{\Delta_H (\xi)}{\Delta_G (\xi)}\rho(x).
\]
\end{proposition}

\begin{theorem}
Given any function $\rho$ as above for $(G:H)$, there is a quasi-invariant measure
$\lambda$ on $G/H$ such that
\[
\frac{d\lambda_x}{d\lambda}(yH) = \frac{\rho(xy)}{\rho(y)}.
\]
($x,y\in G$).
\end{theorem}
\end{document}
