\documentclass[landscape]{slides}
\usepackage{amsmath, amsthm, float, def, enumerate, amsfonts, latexsym, amssymb}
\usepackage{wasysym}
% Try kdvi inverse search feature!
%\usepackage[active]{srcltx}

%\setlength{\partopsep}{-10 pt}\setlength{\topsep}{-10 pt}

\title{Actions of Semitopological Semigroups on Hausdorff Spaces}
\author{Alan Kydd}
\date{December 9, 2004}
\pagestyle{plain}

\newtheorem{defn}{Definition}
\newtheorem{theorem}[defn]{Theorem}
\newtheorem{proposition}[defn]{Proposition}
\newtheorem{corollary}[defn]{Corollary}
\newtheorem{lemma}[defn]{Lemma}

\begin{document}

\maketitle

%\begin{slide}
%\textbf{Introduction}\\
%A semigroup $G$ together with a Hausdorff topology for which multiplication
%in $G$ is separately [resp. jointly] continuous is called a semitopological
%[resp. topological] semigroup.
%A jointly continuous action of $G$ on a topological Hausdorff space $X$ is
%a jointly continuous map $(a,x) \mapsto ax$ from $G\times X$ into $X$, such that
%$(ab)x = a(bx)$ for all $a$, $b\in G$ and $x\in X$, and such that the map 
%$x\mapsto ax$, from $X$ 
%into $X$, is continuous for each $a\in G$.

%In the first half of this thesis we demonstrate that
%when $G$ has a continuous left action on $X$, several common
%subspaces of $CB(G)$ have analogues in $CB(X)$ which preserve
%many of the properties of the original subspaces.
%We define an action of the dual space of norm closed subspaces of 
%$CB(G)$,
%on the dual space of norm closed subspaces
%of $CB(X)$.  We then consider the topological center of the action
%of $LUC(G)^*$ on $LUC(G,X)^*$.
%As an application of this action,
%we examine the relationship between almost $G$-periodic points,
%$G$-minimal sets, and elements of $\beta X$ that are in the
%support of a $G$-invariant measure for $\beta X$.

%The last chapter of this thesis investigates fixed point properties
%of the action of a locally compact group $G$ on a locally convex 
%topological space $X$.
%We use Day's fixed point theorem to show the existence of an invariant
%measure on coset spaces of certain groups.
%When $H$ is a closed subgroup of $G$, we define a fixed point property 
%for the pair $(G:H)$ such that the
%action of $G$ on $X$ need not be jointly continuous.
%We show that this fixed point property is
%equivalent to a stronger fixed point property for $(G:H)$, defined by Eymard.

%\end{slide}

\begin{slide}
{\large\textbf{Table of Contents}}
\begin{list}{}{\itemsep -8 pt}
\item[1.] Notation and Definitions
\item[2.] Analysis on $G$-invariant spaces
\begin{list}{}{\itemsep -8 pt \topsep -10 pt}
\item[a.] Definitions
\item[b.] $G$-invariant function spaces
\item[c.] Arens action of $LUC(G)^*$ on $LUC(G,X)^*$
\item[d.] $G$-minimal sets and $G$-invariant measures of $\beta X$
\end{list}
\item[3.] Fixed point properties
\begin{list}{}{\itemsep -8 pt \topsep -10 pt}
\item[a.] Existence of a $G$-invariant measure on coset spaces
\item[b.] A Fixed Point Property for the pair $(G:H)$
\end{list}
\end{list}
\end{slide}

\begin{slide}
{\large\bf{Notation and Definitions}}

{\small
A {\it semigroup} is a nonempty set $G$ together with an associative binary operation,
which we usually call multiplication, from $G\times G$ into $G$.

A semigroup $G$ together with a Hausdorff topology for which
multiplication in $G$ is continuous from the left is called a 
{\it right topological semigroup}.

A semigroup $G$ together with a Hausdorff topology for which
multiplication in $G$ is separately [resp. jointly] continuous is called a
{\it semitopological} [resp. {\it topological}\,] {\it semigroup}.
When the topology of $G$ is locally compact, $G$ is called a {\it locally compact semitopological}
[resp. {\it topological}\,] {\it semigroup}.

A group $G$ equipped with a Hausdorff topology is called a {\it topological group}
if both multiplication and inversion are continuous.  When the topology of such a group is
locally compact, we say $G$ is a {\it locally compact group}.

$G$ shall always denote a semitopological semigroup, unless otherwise stated.
$X$ shall always denote a Hausdorff topological space, and we write $X_d$ when $X$ has the discrete topology.}
\end{slide}

\begin{slide}
%For any nonempty $X$,
Let $X$ be any (nonempty) topological space.
\[
\ell^\infty(X) =\text{ all bounded complex-valued functions on }X.
\]
\[
CB(X) = \text{ all bounded, continuous complex-valued functions on }X.
\]
\begin{align*}
C_0 (X) = \{f\in CB(X): &\text{for every $\epsilon >0$, there exists a compact} \\
&\text{subset $K\subset X$ (depending on $f$ and $\epsilon$)}\\
&\text{such that $|f(x)| < \epsilon$ for every $x\in X\backslash K$}\}.
\end{align*}
\[
C_c (X) = \{f\in CB(X):\text{$f$ has compact support}\}.
\]
\end{slide}

\begin{slide}
$\ell^\infty(X)$ is a commutative unital $C^*$-algebra with respect to the supremum norm
\[
\|f\|_\infty = \sup_{x\in X} |f(x)|,
\]
with pointwise addition and multiplication, and involution
\[
f^*(x) = \ol{f(x)}.
\]

$C_c(X) \subseteq C_0(X) \subseteq CB(X)$, with equality for compact $X$.

Generally we always consider $CB(X)$ to have the
supremum norm topology, noting that $CB(X_d) = \ell^\infty(X)$.
\end{slide}

\begin{slide}
For any $f\in \ell^\infty(G)$ and $a\in G$, we denote the left and right translations
of $f$ by $a$ by
\begin{equation}\label{translations}
\ell_a f(b) = f(ab),
\end{equation}
and
\begin{equation}
r_a f(b) = f(ba),
\end{equation}
for all $b\in G$.
We denote the left and right orbits of $f$ by
\[
O_\ell f = \{ \ell_a f: a\in G \},
\]
and
\[
O_r f = \{ r_a f: a\in G \}.
\]
\end{slide}

\begin{slide}
A linear subspace $Y \subseteq \ell^\infty(G)$ is {\it left} [resp. {\it right}\,] {\it invariant} if
$O_\ell f \subseteq Y$ [resp. $O_r f \subseteq Y$] for all $f\in Y$.  $Y$ is {\it invariant}
if it is left and right invariant.

Let $Y$ be a left [resp. right] invariant norm closed subspace of $\ell^\infty(G)$,
and let $Y^*$ denote the dual space of $Y$.  $Y$ is {\it left} [resp. {\it right}\,] {\it introverted}
if the function
\begin{equation}\label{introversions}
n_\ell f(g) = n(\ell_g f)\qquad \text{[resp. }n_r f(g) = n(r_g f)\text{]}
\end{equation}
is in $Y$ for each $n\in Y^*$.
$Y$ is {\it introverted} if it is left and right introverted.
\end{slide}

\begin{slide}
%{\small
Let $X$ be an arbitrary set, let $Y$ be a norm closed subspace of $\ell^\infty(X)$
containing the constant functions.
Denote the identity function in $Y$ by $1_Y$.
A {\it mean} on $Y$ is any $m\in Y^*$ such that $m(1_Y) = 1 = \|m\|$.
We denote all means on $Y$ by $M(Y)$.

Let $Y$ be a left [resp. right] invariant norm closed subspace of $\ell^\infty(G)$ containing the constant
functions.  A mean $m\in Y^*$ is a {\it left} [resp {\it right}\,] {\it invariant mean}
(LIM [resp. RIM]) if $m(\ell_g f) = m(f)$ [resp. $m(r_g f) = m(f)$] for all $g\in G$
and $f\in Y$. We denote the set of all such means by $LIM(Y)$ [resp. $RIM(Y)$].
$m$ is an {\it invariant mean} (IM) if $m$ is both a LIM and a RIM

$G$ is said be to {\it left} [resp. {\it right}\,] {\it amenable} if there exists a LIM [resp. RIM] on $\ell^\infty(G)$.
$G$ is {\it amenable} if there exists an IM on $\ell^\infty(G)$.
A topological group $G$ is said to be
{\it amenable as discrete} if there exists a LIM or a RIM for $\ell^\infty(G_d)$.
%}
\end{slide}

\begin{slide}
Let $Y$ be a left introverted subspace of $\ell^\infty(G)$.
The {\it Arens product} is a map from $Y^* \times Y^*$ to $Y^*$ defined by:
\[
(m,n) \mapsto m\oodot n,
\]
where
\[
m\oodot n(f) = \langle m, n_\ell f\rangle
\]
for all $m$, $n\in Y^*$ and $f\in Y$.
%Also,
%\[
%\|m\oodot n\| \leq \|m\|\|n\|.
%\]

When $Y$ is a right introverted subspace of $\ell^\infty(G)$,
the {\it second Arens product} is defined for $Y^*$ by
\[
(m,n) \mapsto m\,\Box\,n,
\]
where
\[
m\,\Box\,n(f) = \langle m, n_r f\rangle.
\]
\end{slide}

\begin{slide}
Both products are associative, distributive and weak*-weak* continuous
in the first variable.  Furthermore, $Y^*$ is a Banach algebra under either product.

If $Y$ is a left [resp. right] introverted subspace of $\ell^\infty(G)$, then
$Y^*$ is a right topological semigroup under the first [resp. second] Arens product.

Let $Y$ be left $G$ introverted.  The {\it topological center} of $Y^*$ with respect
to the first Arens product is
\[
Z_t(Y) = \{m\in Y^*:\text{ the map }n\mapsto m\oodot n\text{ is w*-w* continuous on }Y^*\}.
\]

Let $Y$ be introverted.  The {\it topological center} of $Y^*$ is defined to be
\[
Z_t(Y) = \{m\in Y^*:m\oodot n = m\,\Box\,n\text{ for all }n\in Y^*\}.
\]
$Y$ is called {\it Arens regular} when $Z_t(Y) = Y^*$.
\end{slide}

\begin{slide}
{\large\bf{Analysis on $G$-invariant subspaces}}\\
{\bf Definitions}\\
\begin{defn}
A jointly continuous action of $G$ on a topological Hausdorff space $X$ is
a jointly continuous  map $(a,x) \mapsto ax$ from $G\times X$ into $X$, such that
$(ab)x = a(bx)$ for all $a$, $b\in G$ and $x\in X$, and such that the map 
$x\mapsto ax$, from $X$ 
into $X$, is continuous for each $a\in G$.
\end{defn}
{\bf Example} Multiplication in a topological semigroup $S$
can be viewed as a jointly continuous action of $S$ on itself.

\end{slide}

\begin{slide}
Let $G$ have a jointly continuous action on $X$.\\
\begin{defn}
We denote the left translation of a function\linebreak $f\in \ell^\infty (X)$ by an element $a\in G$ by
\[
%\begin{equation}\label{left}
\lambda_a f(x) = f(ax)
%\end{equation}
\]
for all $x\in X$.
We define an analogue to the right translation (defined
in (\ref{translations})): for $f\in \ell^\infty(X)$ and $x\in X$, define the function
\[
%\begin{equation}\label{right}
\rho_x f(a) = f(ax),
%\end{equation}
\]
for all $a\in G$.\\
\end{defn}
\begin{defn}
We define the left and right orbits of $f\in \ell^\infty(X)$ with respect to $G$ by
\[
O_\lambda f = \{ \lambda_a f: a\in G \},\qquad O_\rho f = \{ \rho_x f: x\in X \}.
\]
\end{defn}
\end{slide}

\begin{slide}
\begin{defn}
Consider a subspace of $CB(X)$ that is defined purely in terms of the action of $G$ on $X$
and the topologies of $G$ and $X$ (i.e. defined independently of the algebraic structure of $X$, if any).
Hereafter, we will always write such a subspace as $Y(G,X)$.  We define the
{\it incident space of $Y(G,X)$} to be $Y(G,G)$, the subspace
% other names for this space?  mutual/intermutial/paired/parallel/incident/twin
of $CB(G)$ defined by the same conditions of $Y(G,X)$, with $X$ replaced by $G$.
For convenience, we write $Y(G)$ for $Y(G,G)$.\\
\end{defn}

\begin{defn}
A linear subspace $Y \subseteq \ell^\infty(X)$ is {\it left $G$ invariant} if
$O_\lambda f \subseteq Y$ for all $f\in Y$.
A subspace $Y(G,X)$ of $CB(X)$ is {\it right $X$ invariant} if
$O_\rho f \subseteq Y(G)$ for all $f\in Y(G,X)$.
$Y(G,X)$ is {\it invariant} if it is left $G$ and right $X$ invariant.\\
\end{defn}
\end{slide}

\begin{slide}
\begin{defn}
Let $Y(G,X)$ be a norm closed, left $G$ invariant subspace of $CB(X)$, and
let $n\in Y(G,X)^*$.
$Y(G,X)$ is {\it left $G$ introverted}
if the function
\[
%\begin{equation}\label{left-i}
n_\lambda f(g) = n(\lambda_g f)
%\end{equation}
\]
is in $Y(G)$ for all $f\in Y(G,X)$.
When $Y(G,X)$ is a norm closed, right $X$ invariant subspace and $m\in Y(G)^*$,
$Y(G,X)$ is {\it right $X$ introverted}
if the function
\[
%\begin{equation}\label{right-i}
m_\rho f(x) = m(\rho_x f)
%\end{equation}
\]
is in $Y(G,X)$ for all $f\in Y(G,X)$.
$Y(G,X)$ is {\it introverted} if it is left $G$ and right $X$ introverted.
\end{defn}
\end{slide}

\begin{slide}
\begin{defn}[Greenleaf]\label{aa}
Let $Y$ be a left $G$-invariant, norm closed subspace of $\ell^\infty(X)$ containing the constant
functions.  A mean $m\in Y^*$ is said to be a {\it left $G$-invariant mean} [GLIM] if
$m(\lambda_g f) = m(f)$ for all $g\in G$ and $f\in Y$.  We denote the set of all such means by $GLIM(Y)$.
$X$ is said to be {\it left amenable} if there exists a GLIM on $\ell^\infty(X)$.
\end{defn}
\end{slide}

\begin{slide}
\begin{defn}
Let $Y_1(G,X)$ be a left introverted subspace of $CB(X)$.
For $m\in Y_1(G)^*$, $n\in Y_1(G,X)^*$ and $f\in Y_1(G,X)$, the {\it left Arens action}
of $Y_1(G)^*$ on $Y_1(G,X)^*$ is defined by the map
\[
(m,n) \mapsto m\odot n,
\]
where
\[
m\odot n(f) = \langle m, n_\lambda f\rangle.
\]
Let $Y_2(G,X)$ be a right introverted subspace of $CB(X)$, let
$m\in Y_2(G)^*$, $n\in Y_2(G,X)^*$ and $f\in Y_2(G,X)$.  We similarly define
the {\it right Arens action} of $Y_2(G)^*$ on $Y_2(G,X)^*$ by
\[
n\dotbox m(f) = \langle n, m_\rho f\rangle.
\]
\end{defn}
\end{slide}

\begin{slide}
\begin{proposition}\label{arens:G-inv}
Let $Y(G,X)$ be a left introverted subspace of $CB(X)$.
The left Arens action of $Y(G)^*$ on $Y(G,X)^*$ has the following
properties:
\begin{list}{}{\itemsep -8pt \topsep -5pt}
\item[(i)] If $n\in GLIM(Y(G,X))$, then for every $m\in M(Y(G))$,\linebreak $m\odot n = n$.
\item[(ii)] $Y(G,X)^*$ is a left Banach-$Y(G)^*$ module.
\item[(iii)] For any $g\in G$ and $x\in X$, $\delta_g\odot\delta_x = \delta_{gx}$.\\
\end{list}
\end{proposition}

\begin{defn}
Let $Y(G,X)$ be a left introverted subspace of $CB(X)$.
We define the {\it topological center} of $Y(G,X)^*$ to be the set
\[
Z_Y = \{m\in Y(G)^*:n\mapsto m\odot n\text{ is w*-w* continuous on }Y(G,X)^*\}.
\]
\end{defn}
\end{slide}

\begin{slide}
\bf{$G$-invariant function spaces}\\
\begin{defn}
A function $f\in CB(X)$ is {\it almost periodic} 
if $O_\lambda f$ is relatively compact in the norm topology of $CB(X)$
(equivalently, if $O_\rho f$ is relatively compact in the norm topology
of $CB(G)$).  We
denote the space of all such functions $AP(G,X)$.\\
\end{defn}


\begin{defn}
A function $f\in CB(X)$ is {\it weakly almost periodic} 
if $O_\lambda f$ is relatively compact in the weak topology of $CB(X)$
(equivalently, if $O_\rho f$ is relatively compact in the weak topology
of $CB(G)$).  We
denote the space of all such functions $WAP(G,X)$.\\
\end{defn}


\begin{defn}
A function $f\in CB(X)$ is called {\it left uniformly continuous} if the
map $a\mapsto\lambda_a f$ from $G$ into $CB(X)$ is continuous.
We denote the set of all such functions by $LUC(G,X)$.
\end{defn}
\end{slide}

\begin{slide}
\begin{theorem}
$AP(G,X)$ is the largest involution closed, introverted $C^*$-subalgebra of $CB(X)$
containing the constant functions such that $m\odot n = n\dotbox m$, and such that
$(m,n)\mapsto m\odot n$ is jointly continuous on bounded subsets.\\
\end{theorem}

\begin{theorem}
$WAP(G,X)$ is the largest involution closed, invariant, left introverted $C^*$-subalgebra of $CB(X)$
containing the constant functions such that the subalgebra is introverted, $m\odot n = n\dotbox m$, and such that
$(m,n)\mapsto m\odot n$ is separately continuous.\\
\end{theorem}

\begin{theorem}
$LUC(G,X)$ is an invariant left introverted $C^*$-subalgebra of $CB(X)$ containing the constant
functions.
\end{theorem}
\end{slide}

\begin{slide}
\begin{itemize}
\item $AP(G,X) \subset WAP(G,X)$
\item $AP(G,X) \subset LUC(G,X)$
\item If $G$ is a locally compact group, then $WAP(G,X) \subset LUC(G,X)$.
\item $f\in WAP(G,X) \Leftrightarrow \lim_i \lim_j f(g_i x_j) = \lim_j \lim_i f(g_i x_j)$ for sequences $\{g_i\}\subset G$ and $\{x_j\}\subset X$,  whenever all limits exist.
\end{itemize}
\end{slide}

\begin{slide}
{\bf Arens action of the Banach algebra $LUC(G)^*$ on $LUC(G,X)^*$}

Using ideas of Lau and Wong, we prove that the measure algebra
$\mathcal{M}(G)$ is a subset of $Z_{LUC}$.

Let $Z$ denote the set of all $m\in LUC(G)^*$ such that the function $m_\rho f$ is in $LUC(G,X)$
for all $f\in LUC(G,X)$, with $m\odot n = n\dotbox m$ for all $n\in LUC(G,X)^*$.\\

\begin{lemma}\label{lau:arens}
Let $m \in LUC(G)^*$.  The following are equivalent:
\begin{list}{}{\itemsep -8pt \topsep -8pt}
\item[(i)] $m\in Z$.
%\item The map $n\mapsto m\odot n$ is weak*-weak* continuous.
\item[(ii)] $m\in Z_{LUC}$.
\item[(iii)] The map $n\mapsto m\odot n$ is weak*-weak* continuous on norm bounded subsets of $LUC(G,X)^*$.
\end{list}
\end{lemma}
\end{slide}

\begin{slide}
For any locally compact space $X$, let $\tau_X$ denote the locally convex topology on $\mathcal{M}(X)$
determined by the family of seminorms $\{p_f : f\in LUC(G,X)\}$, where
$p_f (\mu) =| \int f\:d\mu|$, $\mu\in \mathcal{M}(X)$.\\

\begin{lemma}\label{wong}
Let $G$ be a locally compact semitopological semigroup with jointly continuous action
on a locally compact Hausdorff space $X$.
\begin{list}{}{\itemsep -6pt \topsep -6pt}
\item[(i)] For every $\mu\in \mathcal{M}(G)$, the map $n \mapsto {\mu}\odot n$ is weak*-weak* continuous on norm bounded subsets of $LUC(G,X)^*$.
\item[(ii)] For every $n\in LUC(G,X)^*$, the map $\mu\mapsto {\mu}\odot n$ is $\tau_G$-weak* continuous.
\item[(iii)] For $\mu\in \mathcal{M}(G)$, $\nu\in \mathcal{M}(X)$, ${\mu}\odot{\nu}(f) = \langle\mu * \nu, f\rangle$ for all $f\in C_0(X)$.
\item[(iv)] $n\in GLIM(LUC(G,X))$ if and only if ${\mu}\odot n = n$ for any $\mu\in \mathcal{M}_0 (G) = \{\mu\in \mathcal{M}(G) : \mu \geq 0, \|\mu\| = 1\}$.
%\item $LUC(G,X)$ has a $G$-invariant mean iff there is a net $\{\nu_\alpha \}$ in $M_0 (X)$ such that
%$\mu * \nu_\alpha - \nu_\alpha \rightarrow 0$ in the $\tau$ topology of $M(X)$ for any $\mu \in M_0 (G)$.
\end{list}
\end{lemma}
\end{slide}

\begin{slide}
{\bf Main Result}\\
\begin{theorem}\label{wong2}
Let $G$ be a locally compact semitopological semigroup with jointly continuous action
on a locally compact Hausdorff space X. If $\mu \in \mathcal{M}(G)$, then ${\mu} \in Z_{LUC}$.
\end{theorem}
\proof
By Lemma~\ref{wong} (i), the map ${\mu} \mapsto m\odot {\mu}$ is weak*-weak* continuous on norm bounded subsets
of $LUC(G,X)^*$.  By Lemma~\ref{lau:arens}, ${\mu} \in Z_{LUC}$.
\done
\end{slide}

\begin{slide}
{\bf Remarks}

In \emph{Continuity of {A}rens multiplication on the dual space of
  bounded uniformly continuous functions on locally compact groups and
  topological semigroups} (Math. Proc. Cambridge Philos. Soc. \textbf{99}
  (1986), 273--283), Lau
proved that if $G$ is either a locally compact group or a cancellative discrete semigroup
and $X=G$, then $Z_{LUC} = \mathcal{M}(G)$.

Recently, Neufang, in his paper \emph{On a unified approach to the topological center problem for
  certain {B}anach algebras arising in abstract harmonic analysis} (Arch. Math.
  (Basel) \textbf{82} (2004), no.~2, 164--171),
used a different technique to prove the same result when $G$ is a locally compact group.
\end{slide}

\begin{slide}
{\bf $G$-minimal sets and $G$-invariant measures for $\beta X$}

Let $G$, $X$ be discrete.  $(\kappa, \beta X) =$ the Stone-\u{C}ech compactification
of $X$.  We identify $\kappa (x) \in \beta X$ with $\delta_x \in \ell^\infty (X)^*$.

The left action of $G$ on $X$ extends to an action of $G$ on $\beta X$:
\[
(a, n) \mapsto \kappa (a) \odot n.
\]
Notation:\\
For fixed $n\in \beta X$,
$\kappa(G) \odot n = \{\kappa(g) \odot n : g\in G\}$.\\
For fixed $g\in G$ and any $U\subset \beta X$,
$\kappa(g) \odot U = \{\kappa(g) \odot n: n\in U\}$.
If $K\subset G$ and $U\subset \beta X$,
\[
\{\kappa(K)\}^{-1} \odot U = \{ n\in \beta X : \kappa(k) \odot n \in U\text{ for some }k\in K\}.
\]
If $A\subset G$ and $K$ is as above,
\[
\{K\}^{-1}A = \{g\in G: kg\in A\text{ for some }k\in K\}.
\]
\end{slide}

\begin{slide}
We have an isometric $*$-isomorphism $T$ from $C_c (\beta X) = CB(\beta X)$ onto $\ell^\infty (X)$,
$\tilde{f}\mapsto f$, where
\[
f(x) = \tilde{f}(\kappa(x)),
\]
for $\tilde{f}\in C_c (\beta X)$ and $x\in X$.  We identify
$CB(\beta X)^*$ with $\mathcal{M}(\beta X)$ in the usual way:
\[
\langle T^*n, \tilde{f}\rangle = \int\tilde{f}\:d(T^*n).\\
\]


\begin{proposition}\label{probmeas}
$n\in GLIM(\ell^\infty(X))$ if and only if
$T^* n$ is a probability measure on $\beta X$ such that
$(T^*n)(\{\kappa(g)\}^{-1} \odot U) = (T^*n)(U)$ for all $g\in G$ and Borel sets $U\subset \beta X$.\\
\end{proposition}

\begin{defn}
Let $n\in \ell^\infty(X)^*$.  $T^*n$ is called $G$-{\it invariant} if $n \in GLIM(\ell^\infty(X))$.
\end{defn}
\end{slide}

\begin{slide}
\begin{defn}
$n \in \beta X$ is called {\it left almost $G$-periodic} if for every neighbourhood $U$ of $n$
there exists a subset $A \subset G$ such that there is a finite subset $K \subset G$ with
$G = \{K\}^{-1}A$ and $\kappa(A) \odot n \subset U$.  We denote the set of all almost
$G$-periodic elements in $\beta X$ by $A^{G,X}$.\\
\end{defn}

\begin{defn}
A nonempty subset $U$ of $\beta X$ is called $G$-{\it invariant} if $\kappa(g) \odot U \subset U$ for all
$g\in G$.  $U$ is called $G$-{\it minimal} if it is closed and minimal with respect to this property.
We denote the elements of $\beta X$ which belong to a $G$-minimal set by $B^{G,X}$.
\end{defn}

We denote by $K^{G,X}$ the elements in $\beta X$ which are in the support of some $G$-invariant
measure.\\

%\begin{example}
%Take $n\in \beta X$.  Clearly, for all $a\in G$,
%$\kappa(a) \odot \kappa(G) \odot n \subset \kappa(G) \odot n$.  Thus, $\ol{\kappa(G) \odot n}$
%is a closed $G$-invariant set.
%\end{example}
\end{slide}

\begin{slide}
\begin{proposition}\label{fairchild2.1}
Let $n\in GLIM(\ell^\infty(X))$.  Then $\supp (T^*n)$ is a $G$-invariant set.\\
\end{proposition}

\begin{proposition}~\label{w&w}
Let $n\in M(\ell^\infty(X))$, and let $U$ be a closed subset of $\beta X$.
Then $\supp(T^*n) \subset U$ if and only if $n\in\ol{\conv U}$.
\end{proposition}
{\bf Main Results}\\
\begin{theorem}\label{fairchild3.1}
$A^{G,X} = B^{G,X}$.\\
\end{theorem}

\begin{corollary}
When $G$ acts amenably on $X$, $A^{G,X} \subset K^{G,X}$.
\end{corollary}
\end{slide}

\begin{slide}
We now find conditions on $X$ and $G$ that imply $K^{G,X}\backslash A^{G,X} \neq \emptyset$.
We need the following definition:\\

\begin{defn}
For $A \subset X$, let
\[
d(A) = \sup\:\{ m(\chi_A) : m\in GLIM(\ell^\infty(X)) \}.
\]
$A$ is called a {\it C-subset} for the pair  $(G,X)$ if $d(A) > 0$ and $d(\{K\}^{-1}A) < 1$ for all
finite $K\subset G$.\\
\end{defn}

%\begin{lemma}\label{fairchild3.3}
%Let $A \subset X$, $g\in G$.  Then
%\[
%\{\kappa(g)\}^{-1} \odot \ol{\kappa(A)} = \ol{ \{\kappa(g)\}^{-1} \odot \kappa(A)} = \ol{\kappa(\{g\}^{-1}A)}.\\
%\]
%\end{lemma}

%\begin{theorem}\label{fairchild2.0}
%If left amenable $G$ acts on $X$ and $U$ is a closed $G$-invariant
%subset of $\beta X$, then for each $n\in U$ and $m\in LIM(\ell^\infty(G))$, $T^*(m\odot n)$
%is a $G$-invariant measure with $\supp (T^*(m\odot n)) \subset U$.
%\end{theorem}
%\end{slide}

%\begin{slide}

\begin{theorem}
If $G$ is left amenable and $(G,X)$ has a C-subset $A$, then $\ol{\kappa(A)}\cap\ol{A^{G,X}} = \emptyset$ and
$\ol{\kappa(A)}\cap K^{G,X} \neq \emptyset$.  Thus $A^{G,X} \subsetneq K^{G,X}$.\\
\end{theorem}

\begin{proposition}
Let $G$ be left amenable.  Suppose that the pair $(G,X)$ has no C-subsets.
Then $K^{G,X} \subset \ol{A^{G,X}}$.
\end{proposition}
\end{slide}

\begin{slide}
{\large \bf{Fixed Point Properties}}\\
Let $G$ be a locally compact group, let $H$ be a closed subgroup of $G$.
The coset space $G/H = \{xH: x\in G\}$.  The space $G/H$ admits a
quasi-invariant measure $\lambda$.
\[
L^\infty(G/H) = \text{ ess. bdd. $\lambda$-measurable $\mathbb{C}$-valued functions on }G/H,
\]
with norm
\begin{align*}
\|f\| &= \text{ess.}\sup_{x\in G} |f(x)|\\
&= \inf\{\alpha>0:\{g\in G:|f(g)| > \alpha\}\text{ is locally null}\}\}.
\end{align*}


An affine transformation from a vector space $V$ to itself is a map $T$
such that
\[
T(\alpha x + (1-\alpha)y) = \alpha T(x) + (1-\alpha)T(y)
\]
for all $x,y\in V$ and scalars $\alpha$.
\end{slide}

\begin{slide}
{\bf Existence of a $G$-invariant measure on coset spaces}

We prove, using Day's fixed point theorem in
\emph{Fixed-point theorems for compact convex sets} (Illinois J. Math.
  \textbf{5} (1961), 585--589),
that $G/H$ admits a regular Borel measure $\mu$ on the coset space $G/H$ such that $\mu(gE) = 
\mu(E)$ for all $g \in G$ and all Borel sets $E$ of $G/H$.  The proof uses an idea
of Izzo in the proof of the existence of the Haar measure on locally compact
Abelian groups using the Markov-Kakutani fixed point theorem, found in
\emph{On certain actions of semi-groups on ${L}$-spaces} (Studia
  Math. \textbf{29} (1967), 63--77).
\end{slide}

\begin{slide}
\begin{theorem}[Day's Fixed Point Theorem]
Let $K$ be a compact, convex subset of a locally convex Hausdorff 
topological vector space.  Let $S$ be a semigroup of affine continuous 
transformations of $K$ into itself.  If $S$ is amenable as a discrete 
semigroup, then there exists $k \in K$ such that $Tk=k$ for all $T 
\in S$.\\
\end{theorem}

\begin{lemma}\label{ginv:1}
Let $G$ be a topological group and let $U$ be a symmetric neighbourhood of 
the identity in $G$.  There exists $V \subset G$ such that for 
each $g \in G$, the set $gUU$ contains at least one element of 
$V$, and such that the set $gU$ contains at most one element of $V$.\\
\end{lemma}

\begin{lemma}\label{ginv:2}
Let $X$ be a vector space.
If $K$ is a weak* closed subset of $X^*$ such that for each $x \in X$ the set 
$\{\phi(x):\phi\in K\}$ is bounded, then $K$ is compact.
\end{lemma}
\end{slide}

\begin{slide}
{\bf Main Result}\\
\begin{theorem}\label{coset}
Let $G$ be a locally compact group which is amenable as a discrete group,
and let $H$ be a closed subgroup of $G$.  Then $G/H$ admits a $G$-invariant
measure.
\end{theorem}
\end{slide}

\begin{slide}
\proof
For each $a \in G$, define $T_a: C_c(G/H)^*\to C_c(G/H)^*$ by
\[
\langle T_a,\phi\rangle(f) = \phi(\lambda_a f)
\]
Each $T_a$ is continuous and affine, and $S = \{T_a : a\in G \}$ is a 
representation of $G$.

Fix a symmetric neighbourhood $U$ of the identity in $G$ such that $\ol{U}$ is compact.
Let $K$ be all positive linear functionals $\phi \in C_c(G/H)^*$ which satisfy:
\begin{list}{}{\itemsep -6pt \topsep -5pt}
\item $\phi(f) \leq 1$ for all nonnegative $f \in C_c(G/H)$ that are 
bounded 
above by 1 and supported in $(xU)H = \{(xu)H: u\in U\}$ for some $x \in G$, and
\item $\phi(f) \geq 1$ for all nonnegative $f \in C_c(G/H)$ that are equal 
to 1 on $(xUU)H = \{(xuv)H: u,v\in U\}$ for some $x \in G$.
\end{list}
\end{slide}

\begin{slide}
$K$ is weak* closed and convex.  For each $f \in C_c(G/H)$, the set 
$\{\phi(f):\phi\in K\}$ is bounded.  By Lemma~\ref{ginv:2}, $K$ 
is weak* compact.
Take $V$ as in Lemma~\ref{ginv:1}.
The functional
\[
\psi : f \mapsto \sum_{v\in V} f(vH).
\]
is in $K$.

It follows, from the definition of $K$, that $T_a$ maps $K$ onto itself.
By Day's fixed point theorem, $S = \{T_a: a\in G\}$ has a common 
fixed point in $K$.
\done
\end{slide}

\begin{slide}
{\bf A Fixed Point Property for the pair $(G:H)$}

Eymard defined the following fixed point property for the pair\linebreak$(G:H)$ in
\emph{Moyennes invariantes et repr\'{e}sentations unitaries}
(Lecture Notes in Mathematics, vol. 300, Springer-Verlag, Berlin, 1972).\\
\begin{defn}\label{fpp}
The pair $(G:H)$ is said to have the {\it fixed point property} (FPP) if every jointly continuous
affine action of $G$ on a compact convex subset $K$ of a locally convex
topological space $X$ which has a fixed point for $H$ also has a fixed point for $G$.
\end{defn}
Furthermore, Eymard proved that there exists a GLIM on $L^\infty(G/H)$ if and only if $(G:H)$
has the FPP.
\end{slide}

\begin{slide}
In \emph{A remark on groups with the fixed point property} (Proc. Amer.
  Math. Soc. \textbf{23} (1972), no.~2, 623--624),
Simon considered the following weaker condition of the action of $G$ on $K$.\\
\begin{defn}
A {\it weakly measurable affine action} of $G$ on a compact convex subset $K$ of a locally
convex topological space $X$ is a representation of $G$ by continuous affine maps
$\alpha_g : K \rightarrow K$ such that for each $\phi \in X^*$ and $x\in K$, the map
$g\mapsto \langle\phi, \alpha_g (x)\rangle$ is measurable.
\end{defn}

We define a second fixed point property for the pair $(G:H)$.\\
\begin{defn}
The pair $(G:H)$ is said to have the
FPP2 if every weakly
measurable affine action of $G$ on a compact convex subset $K$ of a locally convex
topological space $X$ which has a fixed point for $H$ also has a fixed point for $G$.
\end{defn}
\end{slide}

\begin{slide}
{\bf Main Result}\\
\begin{theorem}
For $G$ a locally compact group with a closed subgroup $H$, the following are equivalent:
\begin{list}{}{\itemsep -6pt \topsep -8pt}
\item[(i)] There exists a GLIM on $L^\infty (G/H)$.
\item[(ii)] $(G:H)$ has the FPP2.
\item[(iii)] $(G:H)$ has the FPP.
\item[(iv)] There exists a GLIM on $LUC(G,G/H)$.
\end{list}
\end{theorem}

\end{slide}
%%end

\end{document}
